%编译环境:Windows 7+CTeX_2.9.2.164_Full+XeLaTeX
%文档编码:UTF-8
\documentclass[12pt,a4paper,landscape,fancyhdr,fntef,oneside]{ctexbook}
%\usepackage[boldfont,slantfont,CJKaddspaces,CJKchecksingle]{xeCJK}
\setCJKmainfont[BoldFont={STZhongsong}, ItalicFont={Adobe Kaiti Std}]{SimSun}
\setCJKmonofont{Adobe Fangsong Std}
\setCJKfamilyfont{song}{Adobe Song Std}

\setmainfont{Times New Roman}           % 英文缺省字体
\setsansfont{Arial}                                 % 英文无衬线字体
\setmonofont{Courier New}                   % 英文打字机(等宽)字体

\usepackage{verbatim}
\usepackage{fancyvrb}
\usepackage{geometry}
\usepackage{underscore} %处理下划线(如变量名称中带下划线)
\usepackage{amsmath,amsfonts,amssymb}   % Math packages
\usepackage{tikz}
\usepackage{tabularx}
\usepackage[CJKbookmarks, colorlinks, bookmarksnumbered=true,pdfstartview=FitH,linkcolor=black,citecolor=black]{hyperref}

%\usepackage{shortvrb}
%\MakeShortVerb{\|}

\usepackage{graphicx}
% 设置图形文件的搜索路径
\graphicspath{{chapter/}{figure/}}


%\newcommand\cs{
%\begin{center}
%  \S
%\end{center}
%}

\DefineVerbatimEnvironment{code}{Verbatim}%
  {frame=lines,framerule=0.5mm,rulecolor=\color{black},%
  fontseries=tt,xleftmargin=4mm,tabsize=4,numbers=left,numbersep=1.5mm}
\renewenvironment{quote}
                {\kaishu
                 \list{}{\rightmargin   2em
                         \listparindent 2em
                         \itemindent    2em
                         \parsep        0em}
                 \item\relax}
                {\endlist}

\usepackage{listings}
\lstset{
language=C,
extendedchars=false,
frame=lines,%none|leftline|topline|bottomline|lines|single|shadowbox
framerule=0.3mm,
rulesepcolor=\color{red!20!green!20!blue!20},
basicstyle=\ttfamily\footnotesize,
showstringspaces=false,
xleftmargin=2em,
xrightmargin=2em
}

\CTEXsetup[name={教程,}]{chapter}
\CTEXsetup[number={\arabic{chapter}}]{chapter}
\setcounter{chapter}{0}%-1:从0开始编号
\renewcommand\thesection{\arabic{section}}
\CTEXsetup[format+={\flushleft}]{section}

%%%%% Definitions
%%% Define a new command that prints the title only
\makeatletter                           % Begin definition
\def\printtitle{%                       % Define command: \printauthor
    {\huge {\@title}\par}}      % Typesetting
\makeatother                            % End definition

\title{
\hfill{\small \textbf{How to Write a Video Player in
Less Than 1000 Lines}}\\
\hfill \textbf{An ffmpeg and SDL Tutorial}\\
\hfill \textbf{如何用FFmpeg编写一个简单播放器}\\
\hfill {\large\kaishu 中文版}}

%%% Define a new command that prints the author(s) only
\makeatletter                           % Begin definition
\def\printauthor{%                  % Define command: \printtitle
    {\large \@author}}              % Typesetting
\makeatother                            % End definition

\author{
        \hfill Stephen Dranger \quad 著\\
        \hfill 不详\quad 译\\
}

\renewcommand\maketitle{%
  \newpage
  \thispagestyle{empty}
  %%% Top of the page: Author, Title and Abstact

%\begin{minipage}{0.48\linewidth}

    %\vspace{.2\textheight}
    \includegraphics[height=0.1314\textheight]{FFmpegnewlogo.png}\\ %Nekomata.jpg

%\end{minipage} %\hspace{0pt}
%
%\begin{minipage}{0.02\linewidth}
%   \vspace{.2\textheight}
%   \rule{3pt}{0.6\textheight}
%\end{minipage} %\hspace{0pt}
%
\begin{minipage}{\linewidth}
\begin{flushleft}
\vspace{.2\textheight}
\printtitle
\vspace{.2\textheight}
\printauthor
\end{flushleft}
\end{minipage}
%\vspace{20pt}      % Add some vertical spacing to seperate the abstract from the rest of the article
\clearpage}
\renewcommand\contentsname{目\hspace{2em}录}

\begin{document}
\maketitle

\newpage
\thispagestyle{empty}
{\hfil \huge \textbf{前\hspace{2em}言}}\par
\vspace{.1\textheight}

教程英文原文由Stephen Dranger\footnote{dranger@gmail.com} 写成,并发布在http://dranger.com/ffmpeg/ffmpeg.html ,中文版原始翻译出处不详,本文档参考百度文库中的\href{http://wenku.baidu.com/view/2a30ffef0975f46527d3e1ac.html }{翻译版本}及部分其它译本进行校正和重排。正如原文作者在教程开始部分指出的,教程中有些内容已经过时了,教程提供的代码中使用的API 有些已经被FFmpeg 废弃了。不过好在有开发者更新教程的代码\footnote{https://github.com/chelyaev/ffmpeg-tutorial}到比较新的FFmpeg版本\footnote{ffmpeg version N-50314-gf6fff8e},因此文章作为开发者的入门教程还是很有参考价值。

本文档LaTeX源码和PDF文档发布在Github:https://github.com/mengyingchina/FFmpeg-Tutorial-CN ,你可以在这里获取最新更新,如果有发现文字或翻译错误,请提交到\href{https://github.com/mengyingchina/FFmpeg-Tutorial-CN/issues}{FFmpeg-Tutorial-CN/issues} 。

\vspace{1em}
原文的版权声明:
\begin{quote}
\noindent This work is licensed under the Creative Commons Attribution-Share Alike 2.5 License. To view a copy of this license, visit http://creativecommons.org/licenses/by-sa/2.5/ or send a letter to Creative Commons, 543 Howard Street, 5th Floor, San Francisco, California, 94105, USA.

\noindent Code examples are based off of FFplay, Copyright (c) 2003 Fabrice Bellard, and a tutorial by Martin Bohme.
\end{quote}


\frontmatter
\tableofcontents

\mainmatter

\chapter*{写在前面}
\begin{quote}
\textbf{警告:本教程是有点过时了}

到目前为止,需要做的变动只是用sws_scale 函数来替换现已不在使用的 img_convert 函数,在第\ref{ch8}部分中有相关描述。很长一段时间以来我一直想更新这篇文档,所以这是一个很好的可以开始的理由。在过去的3年中我收到了很多有用的建议和答案,如果你发过邮件给我,非常感谢你的贡献! 希望本教程会很快更新。
\end{quote}

FFMPEG 是一个很好的库,可以用来创建视频应用或者生成特定的工具。FFMPEG几乎为你把所有的繁重工作都做了,比如解码、编码、复用和解复用。这使得多媒体应用程序变得容易编写。它是一个简单的,用C编写的,快速的并且能够解码几乎所有你能用到的格式,当然也包括编码多种格式。

唯一的问题是它的文档基本上是没有的。有一个单独的教程讲了它的基本原理另外还有一个使用doxygen 生成的文档。这就是为什么当我决定研究 FFMPEG来弄清楚音视频应用程序是如何工作的过程中,我决定把这个过程用文档的形式记录并且发布出来作为初学教程的原因。

在FFMPEG工程中有一个示例的程序叫作ffplay。它是一个用C编写的利用ffmpeg来实现完整视频播放的简单播放器。这个教程将从原来Martin Bohme写的一个更新版本的教程开始,基于Fabrice Bellard的ffplay,我将从那里开发一个可以使用的视频播放器。在每一个教程中,我将介绍一个或者两个新的思想并且讲解我们如何来实现它。每一个教程都会有一个C源文件,你可以下载,编译并沿着这条思路来自己做。源文件将向你展示一个真正的程序是如何运行,我们如何来调用所有的部件,也将告诉你在这个教程中技术实现的细节并不重要。当我们结束这个教程的时候,我们将有一个少于1000行代码的可以工作的视频播放器。

在写播放器的过程中,我们将使用SDL来输出音频和视频。SDL是一个优秀的跨平台的多媒体库,被用在MPEG播放、模拟器和很多视频游戏中。你将需要下载并安装SDL开发库到你的系统中,以便于编译这个教程中的程序。

这篇教程适用于具有相当编程背景的人。至少至少应该懂得C并且有队列和互斥量等概念。你应当了解基本的多媒体中的像波形一类的概念,但是你不必知道的太多,因为我将在这篇教程中介绍很多这样的概念。

\chapter{制作屏幕录像}
\label{ch1}
\section{概述}
电影文件有很多基本的组成部分。首先,文件本身被称为\textbf{容器}(container),容器的类型决定了信息被存放在文件中的位置。AVI 和 Quicktime 就是容器的例子。接着,你有一组\textbf{流}(streams), 例如,你经常有的是一个音频流和一个视频流。(一个“流”只是一种想像出来的词语,用来表示“一连串的通过时间来串连的数据元素”)。在流中的数据元素被称为\textbf{帧}(frames)。 每个流是由不同的\textbf{编码器}(codec)来编码生成的。编解码器描述了实际的数据是如何被编码COded和解码DECoded的,因此它的名字叫做CODEC。Divx和MP3就是编解码器的例子。接着从流中被读出来的叫做\textbf{包}(packets)。包是一段数据,它包含了一段可以被解码成方便我们最后在应用程序中操作的原始帧的数据。根据我们的目的,每个包包含了完整的帧或者对于音频来说是许多格式的完整帧。

基本上来说,处理视频和音频流是很容易的:

\begin{verbatim}
  1. 从video.avi 文件中打开视频流 video_stream
  2. 从视频流中读取包到帧中
  3. 如果这个帧还不完整,跳到2
  4. 对这个帧进行一些操作
  5. 跳回到2
\end{verbatim}


在这个程序中使用ffmpeg来处理多种媒体是相当容易的,虽然很多程序可能在对帧进行操作的时候非常的复杂。因此在这篇教程中,我们将打开一个文件,读取里面的视频流,而且我们对帧的操作将是把这个帧写到一个PPM文件中。

\section{打开文件}

首先,来看一下我们如何打开一个文件。通过ffmpeg,你必需先初始化这个库。
\begin{lstlisting}
#include <avcodec.h>
#include <avformat.h>
...
int main(int argc, charg *argv[]) {
av_register_all();
\end{lstlisting}


这里注册了所有的文件格式和编解码器的库,所以它们将被自动的使用在被打开的合适格式的文件上。注意你只需要调用av_register_all()一次,因此我们在主函数main()中来调用它。如果你喜欢,也可以只注册特定的格式和编解码器,但是通常你没有必要这样做。

现在我们可以真正的打开文件:
\begin{lstlisting}
AVFormatContext *pFormatCtx;

// Open video file
if(av_open_input_file(&pFormatCtx, argv[1], NULL, 0, NULL)!=0)
  return -1; // Couldn't open file
\end{lstlisting}

我们通过第一个参数来获得文件名。这个函数读取文件的头部并且把信息保存到我们给的AVFormatContext结构体中。最后三个参数用来指定特殊的文件格式,缓冲大小和格式参数,但如果把它们设置为空NULL或者0,libavformat将自动检测这些参数。

这个函数只是检测了文件的头部,所以接着我们需要检查在文件中的流的信息:
\begin{lstlisting}
// Retrieve stream information
if(av_find_stream_info(pFormatCtx)<0)
  return -1; // Couldn't find stream information
\end{lstlisting}

这个函数为pFormatCtx->streams 填充上正确的信息。我们引进一个手工调试的
函数来看一下里面有什么:
\begin{lstlisting}
// Dump information about file onto standard error
dump_format(pFormatCtx, 0, argv[1], 0);
\end{lstlisting}

现在pFormatCtx->streams 仅仅是一组大小为pFormatCtx->nb_streams的指针,所以让我们先跳过它直到我们找到一个视频流。
\begin{lstlisting}
int i;
AVCodecContext *pCodecCtx;

// Find the first video stream
videoStream=-1;
for(i=0; i<pFormatCtx->nb_streams; i++)
  if(pFormatCtx->streams[i]->codec->codec_type==CODEC_TYPE_VIDEO) {
    videoStream=i;
    break;
  }
if(videoStream==-1)
  return -1; // Didn't find a video stream

// Get a pointer to the codec context for the video stream
pCodecCtx=pFormatCtx->streams[videoStream]->codec;
\end{lstlisting}

流中关于编解码器的信息就是被我们叫做“编解码器上下文”(codec context)的东西。这里面包含了流中所使用的关于编解码器的所有信息,现在我们有了一个指向它的指针。但是我们必需要找到真正的编解码器并且打开它:
\begin{lstlisting}
AVCodec *pCodec;

// Find the decoder for the video stream
pCodec=avcodec_find_decoder(pCodecCtx->codec_id);
if(pCodec==NULL) {
  fprintf(stderr, "Unsupported codec!\n");
  return -1; // Codec not found
}
// Open codec
if(avcodec_open(pCodecCtx, pCodec)<0)
  return -1; // Could not open codec
\end{lstlisting}

有些人可能会从旧的教程中记得有两个关于这些代码其它部分:添加CODEC_FLAG_TRUNCATED到pCodecCtx->flags和添加一个hack来粗糙的修正帧率。这两个修正已经不在存在于ffplay.c 中。因此,我必需假设它们不再必要。我们移除了那些代码后还有一个需要指出的不同点:pCodecCtx->time_base 现在已经保存了帧率的信息。time_base是一个结构体,它里面有一个分子和分母(AVRational)。我们使用分数的方式来表示帧率是因为很多编解码器使用非整数的帧率(例如NTSC 使用29.97fps)。

\section{保存数据}

现在我们需要找到一个地方来保存帧:
\begin{lstlisting}
AVFrame *pFrame;

// Allocate video frame
pFrame=avcodec_alloc_frame();
\end{lstlisting}

因为我们准备输出保存24位RGB色的PPM文件,我们必需把帧的格式从原来的转换为RGB。FFMPEG 将为我们做这些转换。在大多数项目中(包括我们的这个)我们都想把原始的帧转换成一个特定的格式。让我们先为转换来申请一帧的内存。
\begin{lstlisting}
// Allocate an AVFrame structure
pFrameRGB=avcodec_alloc_frame();
if(pFrameRGB==NULL)
  return -1;
\end{lstlisting}

即使我们申请了一帧的内存,当转换的时候,我们仍然需要一个地方来放置原始的数据。我们使用avpicture_get_size来获得我们需要的大小,然后手工申请内存空间:

\begin{lstlisting}
uint8_t *buffer;
int numBytes;
// Determine required buffer size and allocate buffer
numBytes=avpicture_get_size(PIX_FMT_RGB24, pCodecCtx->width,
                            pCodecCtx->height);
buffer=(uint8_t *)av_malloc(numBytes*sizeof(uint8_t));
\end{lstlisting}

av_malloc是ffmpeg的malloc,用来实现一个简单的malloc的包装,这样来保证内存地址是对齐的(4字节对齐或者2字节对齐)。它并不能保护你不被内存泄漏,重复释放或者其它malloc的问题所困扰。

现在我们使用avpicture_fill来把帧和我们新申请的内存来结合。关于AVPicture的结构:AVPicture结构体是AVFrame结构体的子集——AVFrame 结构体的开始部分与AVPicture结构体是一样的。

\begin{lstlisting}
// Assign appropriate parts of buffer to image planes in pFrameRGB
// Note that pFrameRGB is an AVFrame, but AVFrame is a superset
// of AVPicture
avpicture_fill((AVPicture *)pFrameRGB, buffer, PIX_FMT_RGB24,
                pCodecCtx->width, pCodecCtx->height);
\end{lstlisting}

最后,我们已经准备好来从流中读取数据了。

\section{读取数据}

我们将要做的是通过读取包来读取整个视频流,然后把它解码成帧,最后转换格式并且保存。
\begin{lstlisting}
int frameFinished;
AVPacket packet;

i=0;
while(av_read_frame(pFormatCtx, &packet)>=0) {
  // Is this a packet from the video stream?
  if(packet.stream_index==videoStream) {
    // Decode video frame
    avcodec_decode_video(pCodecCtx, pFrame, &frameFinished,
                         packet.data, packet.size);

    // Did we get a video frame?
    if(frameFinished) {
    // Convert the image from its native format to RGB
        img_convert((AVPicture *)pFrameRGB, PIX_FMT_RGB24,
            (AVPicture*)pFrame, pCodecCtx->pix_fmt,
            pCodecCtx->width, pCodecCtx->height);

        // Save the frame to disk
        if(++i<=5)
          SaveFrame(pFrameRGB, pCodecCtx->width,
                    pCodecCtx->height, i);
    }
  }

  // Free the packet that was allocated by av_read_frame
  av_free_packet(&packet);
}
\end{lstlisting}

\marginpar{\rule[-15mm]{0.4mm}{15mm}}{\textbf{关于包Packets 的注释:}\small 从技术上讲一个包可以包含部分或者其它的数据,但是ffmpeg的解释器保证了我们得到的包Packets 包含的要么是完整的要么是多个完整的帧。}


这个循环过程是比较简单的:av_read_frame()读取一个包并且把它保存到AVPacket结构体中。注意我们仅仅申请了一个包的结构体——ffmpeg 为我们申请了内部的数据的内存并通过packet.data指针来指向它。这些数据可以在后面通过av_free_packet()来释放。函数avcodec_decode_video()把包转换为帧。然而当解码一个包的时候,我们可能没有得到我们需要的关于帧的信息。因此,当我们得到下一帧的时候,avcodec_decode_video()为我们设置了帧结束标志frameFinished。最后,我们使用img_convert()函数来把帧从原始格式(pCodecCtx->pix_fmt)转换成为RGB格式。要记住,你可以把一个AVFrame结构体的指针转换为AVPicture 结构体的指针。最后,我们把帧和高度宽度信息传递给我们的SaveFrame 函数。

现在我们需要做的是让SaveFrame函数能把RGB信息写入到一个PPM格式的文件中。我们将生成一个简单的PPM格式文件,请相信,它是可以工作的。

\begin{lstlisting}
void SaveFrame(AVFrame *pFrame, int width, int height, int iFrame) {
  FILE *pFile;
  char szFilename[32];
  int  y;

  // Open file
  sprintf(szFilename, "frame%d.ppm", iFrame);
  pFile=fopen(szFilename, "wb");
  if(pFile==NULL)
    return;

  // Write header
  fprintf(pFile, "P6\n%d %d\n255\n", width, height);

  // Write pixel data
  for(y=0; y<height; y++)
    fwrite(pFrame->data[0]+y*pFrame->linesize[0], 1, width*3, pFile);

  // Close file
  fclose(pFile);
}
\end{lstlisting}

我们做了一些标准的文件打开动作,然后写入RGB数据。我们一次向文件写入一行数据。PPM格式文件的是一种包含一长串的RGB数据的文件。如果你了解HTML色彩表示的方式,那么它就类似于把每个像素的颜色头对头的展开,就像\#ff0000\#ff0000....就表示了了个红色的屏幕。(它被保存成二进制方式并且没有分隔符,但是你自己是知道如何分隔的)。文件的头部表示了图像的宽度和高度以及最大的RGB 值的大小。

现在,回顾我们的main()函数。一旦我们开始读取完视频流,我们必需清理一切:
\begin{lstlisting}
// Free the RGB image
av_free(buffer);
av_free(pFrameRGB);

// Free the YUV frame
av_free(pFrame);

// Close the codec
avcodec_close(pCodecCtx);

// Close the video file
av_close_input_file(pFormatCtx);
return 0;
\end{lstlisting}

你会注意到我们使用av_free来释放我们使用avcode_alloc_fram和av_malloc来分配的内存。

就这些!下面,我们将使用Linux 或者其它类似的平台,运行:
\begin{lstlisting}
gcc -o tutorial01 tutorial01.c -lavutil -lavformat -lavcodec -lz -lavutil -lm
\end{lstlisting}

如果你使用的是老版本的ffmpeg,你可以去掉-lavutil 参数:
\begin{lstlisting}
gcc -o tutorial01 tutorial01.c -lavutil -lavformat -lavcodec -lz -lm
\end{lstlisting}

大多数的图像处理函数可以打开PPM 文件。可以使用一些电影文件来进行测试。

\chapter{输出到屏幕}
\label{ch2}
\section{SDL 和视频}
为了在屏幕上显示,我们将使用 SDL。SDL 是 Simple Direct Layer 的缩写。它是一个出色的多媒体库,适用于多平台,并且被用在许多工程中。你可以从它的官方网站的网址 http://www.libsdl.org/上来得到这个库的源代码或者如果有可能的话你可以直接下载开发包到你的操作系统中。在这个教程中需要你编译这个库。(剩下的几个教程也是一样)

SDL库中有许多种方式来在屏幕上绘制图形,而且它有一个特殊的方式来在屏幕上显示图像——这种方式叫做YUV覆盖(overlay)。YUV(从技术上来讲并不叫YUV而是叫做YCbCr)是一种类似于RGB方式的存储原始图像的格式。粗略的讲,Y是亮度分量,U和V是色度分量。(这种格式比RGB 复杂的多,因为很多的颜色信息被丢弃了,而且可以每两个水平Y采样点,有一个U和一个V采样点)。SDL的YUV覆盖使用一组原始的YUV 数据并且在屏幕上显示出它们。它可以允许4种不同的YUV格式,但是其中的YV12是最快的一种。还有一个叫做YUV420P的YUV格式,它和YV12 是一样的,除了U 和V分量的位置被调换了以外。 420意味着它以4:2:0 的比例进行了二次抽样,基本上就意味着1个颜色分量对应着4个亮度分量。所以它的色度信息只有原来的1/4。这是一种节省带宽的好方式,因为人眼感觉不到这种变化。名称中的P表示这种格式是平面的——简单的说就是Y,U和V 分量分别在不同的数组中。FFMPEG可以把图像格式转换为YUV420P,但是现在很多视频流的格式已经是YUV420P 的了或者可以被很容易的转换成YUV420P格式。

于是,我们现在计划把教程1中的SaveFrame()函数替换掉,让它直接输出我们的帧到屏幕上去。但一开始我们必需要先看一下如何使用SDL库。首先我们必需先包含SDL库的头文件并且初始化它。

\begin{lstlisting}
#include <SDL.h>
#include <SDL_thread.h>

if(SDL_Init(SDL_INIT_VIDEO | SDL_INIT_AUDIO | SDL_INIT_TIMER)) {
  fprintf(stderr, "Could not initialize SDL - %s\n", SDL_GetError());
  exit(1);
}
\end{lstlisting}

SDL_Init()函数告诉了SDL库,哪些特性我们将要用到。当然SDL_GetError()是一个用来手工除错的函数。

\section{创建一个显示(Display)}

现在我们需要在屏幕上的一个地方放上一些东西。在SDL中显示图像的基本区域叫做\textbf{surface}。

\begin{lstlisting}
SDL_Surface *screen;

screen = SDL_SetVideoMode(pCodecCtx->width, pCodecCtx->height, 0, 0);
if(!screen) {
  fprintf(stderr, "SDL: could not set video mode - exiting\n");
  exit(1);
}
\end{lstlisting}

这就创建了一个给定高度和宽度的屏幕。下一个选项是屏幕的颜色深度—— 0 表示使用和当前一样的深度。(这个在OS X系统上不能正常工作,原因请看源代码)现在我们在屏幕上来创建一个YUV 覆盖以便于我们输入视频上去:

\begin{lstlisting}
SDL_Overlay     *bmp;

bmp = SDL_CreateYUVOverlay(pCodecCtx->width, pCodecCtx->height,
                           SDL_YV12_OVERLAY, screen);
\end{lstlisting}

正如前面我们所说的,我们使用YV12来显示图像。

\section{显示图像}

前面那些都是很简单的。现在我们需要来显示图像。让我们看一下是如何来处理完成后的帧的。我们将原来对RGB处理的方式,并且替换SaveFrame()为显示到屏幕上的代码。为了显示到屏幕上,为我们的YUV 覆盖建立一个AVPicture结构体并且设置其数据指针和行尺寸(linesize):

\begin{lstlisting}
 if(frameFinished) {
    SDL_LockYUVOverlay(bmp);

    AVPicture pict;
    pict.data[0] = bmp->pixels[0];
    pict.data[1] = bmp->pixels[2];
    pict.data[2] = bmp->pixels[1];

    pict.linesize[0] = bmp->pitches[0];
    pict.linesize[1] = bmp->pitches[2];
    pict.linesize[2] = bmp->pitches[1];

    // Convert the image into YUV format that SDL uses
    img_convert(&pict, PIX_FMT_YUV420P,
                    (AVPicture *)pFrame, pCodecCtx->pix_fmt,
            pCodecCtx->width, pCodecCtx->height);

    SDL_UnlockYUVOverlay(bmp);
  }
\end{lstlisting}

首先,我们锁定这个覆盖,因为我们将要去改写它。这是一个避免以后发生问题的好习惯。正如前面所示的,这个AVPicture结构体有一个数据指针指向一个有4个元素的指针数组。由于我们处理的是YUV420P,所以我们只需要3个通道即只要三组数据。其它的格式可能需要第四个指针来表示alpha通道或者其它参数。行尺寸正如它的名字表示的意义一样。在YUV 覆盖中相同功能的结构体是像素(pixel)和间距(pitch)。 (“间距”是在SDL里用来表示指定行数据宽度的值)。所以我们现在做的是让我们的pict.data 中的三个数组指针指向我们的覆盖,这样当我们写(数据)到pict的时候,实际上是写入到我们的覆盖中,当然要先申请必要的空间。类似的,我们可以直接从覆盖中得到行尺寸信息。像前面一样我们使用img_convert 来把格式转换成PIX_FMT_YUV420P。

\section{绘制图像}

但我们仍然需要告诉SDL 如何来实际显示我们给的数据。我们也会传递一个表明电影位置、应该缩放到什么宽度和高度的矩形参数给SDL 函数。这样,SDL为我们做缩放并且它可以通过使用显卡来进行快速缩放:

\begin{lstlisting}
SDL_Rect rect;

  if(frameFinished) {
    /* ... code ... */
    // Convert the image into YUV format that SDL uses
    img_convert(&pict, PIX_FMT_YUV420P,
                    (AVPicture *)pFrame, pCodecCtx->pix_fmt,
            pCodecCtx->width, pCodecCtx->height);

    SDL_UnlockYUVOverlay(bmp);
    rect.x = 0;
    rect.y = 0;
    rect.w = pCodecCtx->width;
    rect.h = pCodecCtx->height;
    SDL_DisplayYUVOverlay(bmp, &rect);
  }
\end{lstlisting}

现在我们的视频显示出来了!

让我们再花一点时间来看一下SDL另一个特性:它的事件(event)驱动系统。SDL被设置成当你在SDL程序中点击或移动鼠标,或者向它发送一个信号,它都将产生一个\textbf{事件}的驱动方式。如果你的程序想要处理用户输入的话,就检测这些事件。你的程序也可以产生事件并且传递给SDL 事件系统。当使用SDL进行多线程编程的时候,这相当有用,这方面代码我们可以在教程\ref{ch4}中看到。在这个程序中,我们将在处理完包以后就立即轮询事件。现在而言,我们将处理SDL_QUIT事件以便于我们退出:

\begin{lstlisting}
SDL_Event       event;

    av_free_packet(&packet);
    SDL_PollEvent(&event);
    switch(event.type) {
    case SDL_QUIT:
      SDL_Quit();
      exit(0);
      break;
    default:
      break;
    }
\end{lstlisting}

大功告成!让我们去掉旧的冗余代码,开始编译。如果你使用的是Linux或者其变体,使用SDL库进行编译的最好方式为:
\begin{lstlisting}
gcc -o tutorial02 tutorial02.c -lavutil -lavformat -lavcodec -lz -lm\
`sdl-config --cflags --libs`
\end{lstlisting}


这里的sdl-config命令会打印出用于gcc编译的包含正确SDL库的适当参数。为了进行编译,在你自己的平台你可能需要做的有点不同:请查阅一下SDL文档中关于你的系统的那部分。一旦可以编译,就马上运行它。

当运行这个程序的时候会发生什么呢?电影简直跑疯了!实际上,我们只是以我们能从文件中解码帧的最快速度显示了所有的电影的帧。现在我们没有任何代码来计算出我们什么时候需要显示电影的帧。最后(在教程\ref{ch5}),我们将花足够的时间来探讨同步问题。但一开始我们会先忽略这个,因为我们有更加重要的事情要处理:音频!

\chapter{播放音频}
\label{ch3}
\section{音频}
现在我们要来播放音频。SDL也为我们准备了输出音频的方法。函数SDL_OpenAudio()本身就是用来打开音频设备的。它使用一个叫做SDL_AudioSpec 结构体作为参数,这个结构体中包含了我们将要输出的音频的所有信息。

在我们展示如何建立之前,让我们先解释一下电脑是如何处理音频的。数字音频是由一长串的\textbf{样本}(samples)流组成的。每个样本表示音频波形中的一个值。音频按照一个特定的\textbf{采样率}{sample rate}来进行录制,采样率表示以多快的速度来播放这段样本流,它的表示方式为每秒多少次采样。例如22050 和44100的采样率就是电台和CD常用的采样率。此外,大多音频有不只一个通道来表示立体声或者环绕。例如,如果采样是立体声,那么每次的采样数就为2个。当我们从一个电影文件中得到数据的时候,我们不知道我们将得到多少个样本,但是ffmpeg 将不会给我们部分的样本——这意味着它将不会把立体声分割开来。

SDL播放音频的方式是这样的:你先设置音频的选项:采样率(在SDL的结构体中叫做“freq”,代指\textbf{frequency}),通道数和其它的参数,然后我们设置一个回调函数和一些用户数据(userdata)。当开始播放音频的时候,SDL将不断地调用这个回调函数并且要求它来向音频缓冲填入特定的数量的字节。当我们把这些信息放到SDL_AudioSpec 结构体中后,调用函数SDL_OpenAudio() 就会打开音频设备并且给我们返回\emph{另一个}AudioSpec 结构体。(结构体中的)这些音频指标(specs)是我们会\emph{实际}用到的——因为我们无法保证可以得到所要求的指标。

\section{设置音频}

目前先把讲的记住,因为我们实际上还没有任何关于音频流的信息。让我们回过头来看一下我们的代码,看我们是如何找到视频流的,同样我们也可以找到音频流。

\begin{lstlisting}
// Find the first video stream
videoStream=-1;
audioStream=-1;
for(i=0; i < pFormatCtx->nb_streams; i++) {
  if(pFormatCtx->streams[i]->codec->codec_type==CODEC_TYPE_VIDEO
     &&
       videoStream < 0) {
    videoStream=i;
  }
  if(pFormatCtx->streams[i]->codec->codec_type==CODEC_TYPE_AUDIO &&
     audioStream < 0) {
    audioStream=i;
  }
}
if(videoStream==-1)
  return -1; // Didn't find a video stream
if(audioStream==-1)
  return -1;
\end{lstlisting}

从这里我们可以从描述流的AVCodecContext 中得到我们想要的信息,就像我们得到视频流的信息一样。

\begin{lstlisting}
AVCodecContext *aCodecCtx;

aCodecCtx=pFormatCtx->streams[audioStream]->codec;
\end{lstlisting}

包含在编解码上下文中的所有信息正是我们所需要的用来建立音频的信息:

\begin{lstlisting}
wanted_spec.freq = aCodecCtx->sample_rate;
wanted_spec.format = AUDIO_S16SYS;
wanted_spec.channels = aCodecCtx->channels;
wanted_spec.silence = 0;
wanted_spec.samples = SDL_AUDIO_BUFFER_SIZE;
wanted_spec.callback = audio_callback;
wanted_spec.userdata = aCodecCtx;

if(SDL_OpenAudio(&wanted_spec, &spec) < 0) {
  fprintf(stderr, "SDL_OpenAudio: %s\n", SDL_GetError());
  return -1;
}
\end{lstlisting}

让我们浏览一下这些:
\begin{itemize}
  \item freq:前面所讲的采样率

  \item format:告诉SDL我们将要给的格式。“S16SYS”中的S表示有符号的signed,16表示每个样本是16位长的,SYS表示大小端的顺序是与使用的系统相同的。这些格式是由avcodec_decode_audio2为我们给出来的输入音频的格式。

  \item channels:音频的通道数

  \item silence:这是用来表示静音的值。因为音频采样是有符号的,所以0当然就是这个值。

  \item samples:这是当我们想要更多音频的时候,我们想让SDL给出来的音频缓冲区的尺寸。一个比较合适的值在512到8192之间;ffplay使用1024。

  \item callback:这个是我们的回调函数。我们后面将会详细讨论。

  \item userdata:这个是SDL供给回调函数运行的参数。我们将让回调函数得到整个编解码的上下文;你将在后面知道这样做的原因。

\end{itemize}

最后,我们使用SDL_OpenAudio 函数来打开音频。

如果你还记得前面的教程,我们仍然需要打开音频编解码器本身。这是很显然的。

\begin{lstlisting}
AVCodec         *aCodec;

aCodec = avcodec_find_decoder(aCodecCtx->codec_id);
if(!aCodec) {
  fprintf(stderr, "Unsupported codec!\n");
  return -1;
}
avcodec_open(aCodecCtx, aCodec);
\end{lstlisting}
\section{队列}

嗯!现在我们已经准备好从流中取出音频信息。但是我们如何来处理这些信息呢?我们将会不断地从文件中得到这些包,但同时SDL 也将调用回调函数。解决方法为创建一个全局的结构体变量以便于我们从文件中得到的音频包有地方存放同时也保证SDL中的音频回调函数audio_callback能从这个地方得到音频数据。所以我们要做的是创建一个包的\textbf{队列}(queue)。在ffmpeg中有一个叫AVPacketList 的结构体可以帮助我们,这个结构体实际是一串包的链表。下面就是我们的队列结构体:

\begin{lstlisting}
typedef struct PacketQueue {
  AVPacketList *first_pkt, *last_pkt;
  int nb_packets;
  int size;
  SDL_mutex *mutex;
  SDL_cond *cond;
} PacketQueue;
\end{lstlisting}

首先,我们应当指出nb_packets与size是不一样的,size表示我们从packet->size中得到的字节数。你会注意到我们有一个互斥量mutex和一个条件变量cond在结构体里面。这是因为SDL是在一个独立的线程中来进行音频处理的。如果我们没有正确的锁定这个队列,我们有可能把数据搞乱。我们来看这样一个队列实现是什么样的。每一个程序员应当知道如何来创建的一个队列,但是我们将把这部分也来讨论从而可以学习到SDL的函数。

一开始我们先创建一个函数来初始化队列:
\begin{lstlisting}
void packet_queue_init(PacketQueue *q) {
  memset(q, 0, sizeof(PacketQueue));
  q->mutex = SDL_CreateMutex();
  q->cond = SDL_CreateCond();
}
\end{lstlisting}

接着我们再写一个函数来给队列中填入东西:
\begin{lstlisting}
int packet_queue_put(PacketQueue *q, AVPacket *pkt) {

  AVPacketList *pkt1;
  if(av_dup_packet(pkt) < 0) {
    return -1;
  }
  pkt1 = av_malloc(sizeof(AVPacketList));
  if (!pkt1)
    return -1;
  pkt1->pkt = *pkt;
  pkt1->next = NULL;


  SDL_LockMutex(q->mutex);

  if (!q->last_pkt)
    q->first_pkt = pkt1;
  else
    q->last_pkt->next = pkt1;
  q->last_pkt = pkt1;
  q->nb_packets++;
  q->size += pkt1->pkt.size;
  SDL_CondSignal(q->cond);

  SDL_UnlockMutex(q->mutex);
  return 0;
}
\end{lstlisting}

函数SDL_LockMutex()锁定队列的互斥量以便于我们向队列中添加东西,然后函数SDL_CondSignal()通过我们的条件变量为一个接收函数(如果它在等待)发出一个信号来告诉它现在已经有数据了,接着就会解锁互斥量并让队列可以自由访问。

下面是相应的接收函数。注意函数SDL_CondWait()是如何按照我们的要求让函数\textbf{阻塞}(block)的(例如一直等到队列中有数据)。
\begin{lstlisting}
int quit = 0;

static int packet_queue_get(PacketQueue *q, AVPacket *pkt, int block) {
  AVPacketList *pkt1;
  int ret;

  SDL_LockMutex(q->mutex);

  for(;;) {

    if(quit) {
      ret = -1;
      break;
    }

    pkt1 = q->first_pkt;
    if (pkt1) {
      q->first_pkt = pkt1->next;
      if (!q->first_pkt)
        q->last_pkt = NULL;
      q->nb_packets--;
      q->size -= pkt1->pkt.size;
      *pkt = pkt1->pkt;
      av_free(pkt1);
      ret = 1;
      break;
    } else if (!block) {
      ret = 0;
      break;
    } else {
      SDL_CondWait(q->cond, q->mutex);
    }
  }
  SDL_UnlockMutex(q->mutex);
  return ret;
}
\end{lstlisting}

正如你所看到的,我们已经用一个无限循环包装了这个函数以便于我们想用阻塞的方式来得到数据。我们通过使用SDL中的函数SDL_CondWait()来避免无限循环。基本上,所有的CondWait只等待从SDL_CondSignal()函数(或者SDL_CondBroadcast()函数)中发出的信号,然后再继续执行。然而,虽然看起来我们陷入了我们的互斥体中——如果我们一直保持着这个锁,我们的函数将永远无法把数据放入到队列中去!但是,SDL_CondWait()函数也为我们做了解锁互斥量的动作然后才尝试着在得到信号后去重新锁定它。

\section{意外情况}

你们将会注意到我们有一个全局变量quit,我们用它来保证还没有设置程序退出的信号(SDL会自动处理TERM类似的信号)。否则,这个线程将不停地运行直到我们使用kill -9来结束程序。FFMPEG同样也提供了一个函数来进行回调并检查我们是否需要退出一些被阻塞的函数:这个函数就是:url_set_interrupt_cb。

\begin{lstlisting}
int decode_interrupt_cb(void) {
  return quit;
}
...
main() {
...
  url_set_interrupt_cb(decode_interrupt_cb);
...
  SDL_PollEvent(&event);
  switch(event.type) {
  case SDL_QUIT:
    quit = 1;
...
\end{lstlisting}

当然,这仅仅是用来给ffmpeg中的阻塞情况使用的,而不是SDL中的。我们还必需要设置quit 标志为1。

\section{为队列提供包}


剩下的我们唯一需要为队列所做的事就是提供包了:

\begin{lstlisting}
PacketQueue audioq;
main() {
...
  avcodec_open(aCodecCtx, aCodec);

  packet_queue_init(&audioq);
  SDL_PauseAudio(0);
\end{lstlisting}

函数SDL_PauseAudio()让音频设备最终开始工作。如果没有立即供给足够的数据,它会播放静音。我们已经建立好我们的队列,现在我们准备为它提供包。先看一下我们的读取包的循环:

\begin{lstlisting}
while(av_read_frame(pFormatCtx, &packet)>=0) {
  // Is this a packet from the video stream?
  if(packet.stream_index==videoStream) {
    // Decode video frame
    ....
    }
  } else if(packet.stream_index==audioStream) {
    packet_queue_put(&audioq, &packet);
  } else {
    av_free_packet(&packet);
  }
\end{lstlisting}

注意:我们没有在把包放到队列里的时候释放它,我们将在解码后来释放它。

\section{取出包}

我们最后再来让音频回调函数audio_callback从队列中取出包。回调函数的格式必需为void callback(void *userdata, Uint8 *stream, int len),这里的userdata就是我们给SDL的指针,stream是我们要把音频数据写入的缓冲区指针,len是缓冲区的大小。代码如下:

\begin{lstlisting}
void audio_callback(void *userdata, Uint8 *stream, int len) {

  AVCodecContext *aCodecCtx = (AVCodecContext *)userdata;
  int len1, audio_size;

  static uint8_t audio_buf[(AVCODEC_MAX_AUDIO_FRAME_SIZE * 3) / 2];
  static unsigned int audio_buf_size = 0;
  static unsigned int audio_buf_index = 0;

  while(len > 0) {
    if(audio_buf_index >= audio_buf_size) {
      /* We have already sent all our data; get more */
      audio_size = audio_decode_frame(aCodecCtx, audio_buf,
                                      sizeof(audio_buf));
      if(audio_size < 0) {
    /* If error, output silence */
    audio_buf_size = 1024;
    memset(audio_buf, 0, audio_buf_size);
      } else {
    audio_buf_size = audio_size;
      }
      audio_buf_index = 0;
    }
    len1 = audio_buf_size - audio_buf_index;
    if(len1 > len)
      len1 = len;
    memcpy(stream, (uint8_t *)audio_buf + audio_buf_index, len1);
    len -= len1;
    stream += len1;
    audio_buf_index += len1;
  }
}
\end{lstlisting}

这基本上是一个简单的从另外一个我们将要写的audio_decode_frame()函数中获取数据的循环,这个循环把结果写入到中间缓冲区,尝试着向流中写入len字节并且在我们没有足够的数据的时候会获取更多的数据或者当我们有多余数据的时候保存下来为后面使用。这个audio_buf 的大小为1.5倍的音频帧的大小以便于有一个比较好的缓冲,这个音频帧的大小是ffmpeg给出的。

\section{最后解码音频}
让我们看一下解码器的真正部分:audio_decode_frame:
\begin{lstlisting}
int audio_decode_frame(AVCodecContext *aCodecCtx, uint8_t *audio_buf,
                       int buf_size) {

  static AVPacket pkt;
  static uint8_t *audio_pkt_data = NULL;
  static int audio_pkt_size = 0;

  int len1, data_size;

  for(;;) {
    while(audio_pkt_size > 0) {
      data_size = buf_size;
      len1 = avcodec_decode_audio2(aCodecCtx, (int16_t *)audio_buf, &data_size,
                  audio_pkt_data, audio_pkt_size);
      if(len1 < 0) {
    /* if error, skip frame */
    audio_pkt_size = 0;
    break;
      }
      audio_pkt_data += len1;
      audio_pkt_size -= len1;
      if(data_size <= 0) {
    /* No data yet, get more frames */
    continue;
      }
      /* We have data, return it and come back for more later */
      return data_size;
    }
    if(pkt.data)
      av_free_packet(&pkt);

    if(quit) {
      return -1;
    }

    if(packet_queue_get(&audioq, &pkt, 1) < 0) {
      return -1;
    }
    audio_pkt_data = pkt.data;
    audio_pkt_size = pkt.size;
  }
}
\end{lstlisting}

整个过程实际上从函数的尾部开始,在这里我们调用了packet_queue_get()函数。我们从队列中取出包,并且保存它的信息。然后,一旦我们有了可以使用的包,我们就调用函数avcodec_decode_audio2(),它的功能就像它的姐妹函数avcodec_decode_video() 一样,唯一的区别是它的一个包里可能有不止一个音频帧,所以你可能要调用很多次来解码出包中所有的数据。同时也要记住进行指针audio_buf的强制转换,因为SDL给出的是8位整型缓冲指针而ffmpeg给出的数据是16位的整型指针。你应该也会注意到len1和data_size的不同,len1表示解码使用的数据的在包中的大小,data_size表示实际返回的原始音频数据的大小。

\marginpar{\rule[-15mm]{0.4mm}{15mm}}{\textbf{备注:}\small 为什么使用avcodec_decode_audio2()?因为曾经有一个avcodec_decode_audio(),但现在被废弃了。这个新的函数通过data_size变量来算出从audio_buf读取多少数据。}

当我们得到一些数据的时候,我们立刻返回来看一下是否仍然需要从队列中得到更加多的数据或者我们已经完成了。如果我们仍然有更加多的数据要处理,我们把它保存到下一次。如果我们完成了一个包的处理,我们最后要释放它。

就是这样。我们利用主的读取队列循环从文件得到音频并送到队列中,然后被audio_callback函数从队列中读取并处理,最后把数据送给SDL,于是SDL就相当于我们的声卡。让我们继续并且编译:

\begin{lstlisting}
gcc -o tutorial03 tutorial03.c -lavutil -lavformat -lavcodec -lz -lm \
`sdl-config --cflags --libs`
\end{lstlisting}

啊哈!视频虽然还是像原来那样快,但是音频可以正常播放了。这是为什么呢?因为音频信息中带有采样率信息——虽然我们把音频数据尽可能快的填充到声卡缓冲中,但是音频设备却会按照原来流中指定的采样率来进行播放。

我们几乎已经准备好来开始同步音频和视频了,但是首先我们需要一些程序的梳理工作。用队列来处理音频和在独立的线程中播放音频的方法效果很好:它使得程序更加更加易于控制和模块化。在我们开始同步音视频之前,我们需要让我们的代码更加易于处理。所以下次要讲的是:创建线程。

\chapter{创建线程}
\label{ch4}
\section{概述}

上面通过SDL的音频功能添加了音频支持,SDL启动一个线程监听音频回调函数,视频显示也将如此。这使得代码更加模块化,更容易协调,尤其是当我们想要添加同步时。现在从哪开始呢?

首先注意到我们的主函数要处理的太多了:事件循环、读取数据包、解码视频。所以我们要做的是分开它们:创建一个负责解码数据包的线程;然后将数据包添加到队列中,并由相应的音频和视频的线程读取。我们已经这样设置了音频线程;视频线程将会复杂一点,因为我们要自己显示视频。我们将在主循环中添加显示代码。我们将把视频显示和事件循环放在一起,而不是每次循环显示视频。思路是这样的,解码视频,把产生的帧存到\emph{另一个}队列里,然后创建一个自定义事件 (FF_REFRESH_EVENT)并添加到事件系统里。那么当事件循环看到这个事件时,它将显示队列中的下一帧。下面是手绘的 ASCII字符图,指明流程:

\newpage
\begin{verbatim}
 ________ audio  _______      _____
|        | pkts |       |    |     | to spkr
| DECODE |----->| AUDIO |--->| SDL |-->
|________|      |_______|    |_____|
    |  video     _______
    |   pkts    |       |
    +---------->| VIDEO |
 ________       |_______|   _______
|       |          |       |       |
| EVENT |          +------>| VIDEO | to mon.
| LOOP  |----------------->| DISP. |-->
|_______|<---FF_REFRESH----|_______|
\end{verbatim}

移动视频循环到事件里的主要意图是使用SDL_Delay线程,让我们可以控制在屏幕是显示下一帧的确切时间。当我们最终在下一教程中同步视频时,会在其中添加代码,用来控制下一帧的刷新,这样让恰当的图像在恰当的时间显示。

\section{简化代码}

下面我们将清理一下代码。我们有音频和视频编解码器的全部信息,也将添加队列和缓冲区,并且难说不会再添其它什么东西。所有这些都是一个逻辑单元,即电影。我们将创建一个大的结构体,名为VideoState,包含所有的信息。


\begin{lstlisting}
typedef struct VideoState {

  AVFormatContext *pFormatCtx;
  int             videoStream, audioStream;
  AVStream        *audio_st;
  PacketQueue     audioq;
  uint8_t         audio_buf[(AVCODEC_MAX_AUDIO_FRAME_SIZE * 3) / 2];
  unsigned int    audio_buf_size;
  unsigned int    audio_buf_index;
  AVPacket        audio_pkt;
  uint8_t         *audio_pkt_data;
  int             audio_pkt_size;
  AVStream        *video_st;
  PacketQueue     videoq;

  VideoPicture    pictq[VIDEO_PICTURE_QUEUE_SIZE];
  int             pictq_size, pictq_rindex, pictq_windex;
  SDL_mutex       *pictq_mutex;
  SDL_cond        *pictq_cond;

  SDL_Thread      *parse_tid;
  SDL_Thread      *video_tid;

  char            filename[1024];
  int             quit;
} VideoState;
\end{lstlisting}

这里可以看出我们下一步工作的一些眉目。首先,是基础信息,包括格式上下文(format contex),音视频流标志(indices),及相应的AVStream 对象。可以看到,我们把一些音频缓冲区(buffers)信息放到结构体里了。这些(audio_buf、 audio_buf_size 等)都是与存在(或不在)缓冲区里的音频有关的信息\footnote{原文:These (audio_buf, audio_buf_size, etc.) were all for information about audio that was still lying around (or the lack thereof).}。我们为视频添加了另一个队列,和解码帧(存成一个覆盖)缓冲区(它将用作队列,不需要什么花哨的数据结构)。VideoPicture 是我们自定义的结构(当使用它的时侯,你会知道它的含义)。我们也注意到为额外创建的两个线程分配了指针,还有退出标志,电影文件名。

下面回到主函数,看程序里的变化。让我们初始化VideoState结构:

\begin{lstlisting}
int main(int argc, char *argv[]) {

  SDL_Event       event;

  VideoState      *is;

  is = av_mallocz(sizeof(VideoState));
  \end{lstlisting}

av_mallocz()是一个很好的函数,分配内存并初始化为0。

然后我们会初始化显示缓冲区pictq 的锁,因为事件循环会调用显示函数,记住,显示函数将从pictq里获取预解码的帧。同时,我们视频解码器将在pictq里放入它的信息——我们不知道谁将先发生。希望你已经意识到这是一个典型的\textbf{竞态条件}(race condition)。所以开始任何线程之前,先初始化分配它。让我们把电影的文件名复制到VideoState里。

\begin{lstlisting}
pstrcpy(is->filename, sizeof(is->filename), argv[1]);

is->pictq_mutex = SDL_CreateMutex();
is->pictq_cond = SDL_CreateCond();
\end{lstlisting}

av_strlcpy是ffmpeg里添加了边界检查,比strncpy更好的函数。

\section{我们的第一个线程 }

让我们最终启动线程,完成实际的工作:
\begin{lstlisting}
schedule_refresh(is, 40);

is->parse_tid = SDL_CreateThread(decode_thread, is);
if(!is->parse_tid) {
  av_free(is);
  return -1;
}
\end{lstlisting}

我们稍后将会定义schedule_refresh函数。它所做的基本上是告诉系统在指定的毫秒数后发送FF_REFRESH_EVENT事件。当它在实际队列里时,会回调视频刷新函数。让我们先看一下SDL_CreateThread()。

SDL_CreateThread(),可以创建一个和原始线程一样拥有完全访问权限的新线程,并运行传入的函数。在这里,我们调用了decode_thread(),还有VideoState结构作为参数。函数的上半部分没有有什么新的,仅完成打开文件和查找的音视频流索引的工作。唯一不同的是,在我们大结构里保存格式上下文。我们找到了流标志后,将调用另一个我们定义的函数,stream_component_open()。 大事化小是很自然的事,我们做了很多类似的事来设置视频和音频编解码器,我们将把它封装为函数,以重用代码。

我们使用stream_component_open() 函数查找解码器,初始化音频选项,将重要的信息保存到我们大的结构,并启动音频和视频的线程。在这个函数中我们还做了一些其它的,如强制指定而不是自动检测的编解码器种类等。代码如下:

\begin{lstlisting}
int stream_component_open(VideoState *is, int stream_index) {

  AVFormatContext *pFormatCtx = is->pFormatCtx;
  AVCodecContext *codecCtx;
  AVCodec *codec;
  SDL_AudioSpec wanted_spec, spec;

  if(stream_index < 0 || stream_index >= pFormatCtx->nb_streams) {
    return -1;
  }

  // Get a pointer to the codec context for the video stream
  codecCtx = pFormatCtx->streams[stream_index]->codec;

  if(codecCtx->codec_type == CODEC_TYPE_AUDIO) {
    // Set audio settings from codec info
    wanted_spec.freq = codecCtx->sample_rate;
    /* .... */
    wanted_spec.callback = audio_callback;
    wanted_spec.userdata = is;

    if(SDL_OpenAudio(&wanted_spec, &spec) < 0) {
      fprintf(stderr, "SDL_OpenAudio: %s\n", SDL_GetError());
      return -1;
    }
  }
  codec = avcodec_find_decoder(codecCtx->codec_id);
  if(!codec || (avcodec_open(codecCtx, codec) < 0)) {
    fprintf(stderr, "Unsupported codec!\n");
    return -1;
  }

  switch(codecCtx->codec_type) {
  case CODEC_TYPE_AUDIO:
    is->audioStream = stream_index;
    is->audio_st = pFormatCtx->streams[stream_index];
    is->audio_buf_size = 0;
    is->audio_buf_index = 0;
    memset(&is->audio_pkt, 0, sizeof(is->audio_pkt));
    packet_queue_init(&is->audioq);
    SDL_PauseAudio(0);
    break;
  case CODEC_TYPE_VIDEO:
    is->videoStream = stream_index;
    is->video_st = pFormatCtx->streams[stream_index];

    packet_queue_init(&is->videoq);
    is->video_tid = SDL_CreateThread(video_thread, is);
    break;
  default:
    break;
  }
}
\end{lstlisting}

除了现在通用于到音频和视频之外,这和我们之前的代码几乎相同。注意,我们使用大结构作为音频回调的userdata,而不是aCodecCtx。 我们也把流存为audio_st和video_st,并像设置音频队列一样添加了视频队列。该启动视频和音频线程了,代码如下:


\begin{lstlisting}
    SDL_PauseAudio(0);
    break;

/* ...... */

    is->video_tid = SDL_CreateThread(video_thread, is);
\end{lstlisting}

记得我们上次用过SDL_PauseAudio(),SDL_CreateThread()的使用和上次完全一样。我们即将回到video_thread()函数。

在此之前,让我们回到我们的decode_thread()函数的下半。它基本上只是 for 循环,读取数据包中并把它放在正确的队列:
\begin{lstlisting}
  for(;;) {
    if(is->quit) {
      break;
    }
    // seek stuff goes here
    if(is->audioq.size > MAX_AUDIOQ_SIZE ||
       is->videoq.size > MAX_VIDEOQ_SIZE) {
      SDL_Delay(10);
      continue;
    }
    if(av_read_frame(is->pFormatCtx, packet) < 0) {
      if(url_ferror(&pFormatCtx->pb) == 0) {
    SDL_Delay(100); /* no error; wait for user input */
    continue;
      } else {
    break;
      }
    }
    // Is this a packet from the video stream?
    if(packet->stream_index == is->videoStream) {
      packet_queue_put(&is->videoq, packet);
    } else if(packet->stream_index == is->audioStream) {
      packet_queue_put(&is->audioq, packet);
    } else {
      av_free_packet(packet);
    }
  }
\end{lstlisting}

这里没有什么新东西,除了我们给音频和视频队列限定了一个最大值并且我们添加一个检测读错误的函数。格式上下文里面有一个叫做pb的ByteIOContext类型结构体。这个结构体是用来保存一些低级的文件信息。函数url_ferror 用来检测结构体并发现是否有些读取文件错误。

在循环以后,我们的代码是用等待其余的程序结束和提示我们已经结束的。这些代码是有用的,因为它指示出了如何驱动事件——后面我们显示视频会用到。
\begin{lstlisting}
  while(!is->quit) {
    SDL_Delay(100);
  }

 fail:
  if(1){
    SDL_Event event;
    event.type = FF_QUIT_EVENT;
    event.user.data1 = is;
    SDL_PushEvent(&event);
  }
  return 0;
\end{lstlisting}

我们使用SDL 常量SDL_USEREVENT来从用户事件中得到值。第一个用户事件的值应当是SDL_USEREVENT,下一个是 SDL_USEREVENT +1 并且依此类推。在我们的程序中FF_QUIT_EVENT 被定义成SDL_USEREVENT+2。如果喜欢,我们也可以传递用户数据,在这里我们传递的是大结构体的指针。最后我们调用SDL_PushEvent()函数。在我们的事件分支中,我们只是像以前放入SDL_QUIT_EVENT部分一样。我们将在自己的事件队列中详细讨论,现在只是确保我们正确放入了FF_QUIT_EVENT 事件,我们将在后面捕捉到它并且设置我们的退出标志quit。

\section{得到帧:video_thread}
当我们准备好解码器后,我们开始视频线程。这个线程从视频队列中读取包,把它解码成视频帧,然后调用queue_picture 函数把处理好的帧放入到图像队列中:
\begin{lstlisting}
int video_thread(void *arg) {
  VideoState *is = (VideoState *)arg;
  AVPacket pkt1, *packet = &pkt1;
  int len1, frameFinished;
  AVFrame *pFrame;

  pFrame = avcodec_alloc_frame();

  for(;;) {
    if(packet_queue_get(&is->videoq, packet, 1) < 0) {
      // means we quit getting packets
      break;
    }
    // Decode video frame
    len1 = avcodec_decode_video(is->video_st->codec, pFrame, &frameFinished,
                packet->data, packet->size);

    // Did we get a video frame?
    if(frameFinished) {
      if(queue_picture(is, pFrame) < 0) {
    break;
      }
    }
    av_free_packet(packet);
  }
  av_free(pFrame);
  return 0;
}
\end{lstlisting}

这里的很多函数应该很熟悉吧。我们把avcodec_decode_video 函数移到了这里,替换了一些参数,例如:我们把AVStream 保存在我们自己的大结构体中,所以我们可以从那里得到编解码器的信息。我们仅仅是不断地从视频队列中取包一直到有人告诉我们要停止或者出错为止。

\section{把帧队列化}

让我们看一下保存解码后的帧pFrame到图像队列中去的函数。因为我们的图像队列是SDL覆盖的集合(基本上不用让视频显示函数再做计算了),我们需要把帧转换成相应的格式。我们保存到图像队列中的数据是我们自己做的一个结构体。

\begin{lstlisting}
typedef struct VideoPicture {
  SDL_Overlay *bmp;
  int width, height; /* source height & width */
  int allocated;
} VideoPicture;
\end{lstlisting}

我们的大结构体有一个可以保存这些的缓冲区。然而,我们需要自己来申请SDL_Overlay(注意:allocated 标志会指明我们是否已经做了这个申请的动作)。

为了使用这个队列,我们有两个指针——写入指针和读取指针。我们还记录实际有多少图像在缓冲区。要写入到队列中,我们先要等待缓冲清空以便于有位置来保存我们的VideoPicture。然后我们检查我们是否已经申请到了一个可以写入覆盖的索引号。如果没有,我们要申请一段空间。我们也要重新申请缓冲如果窗口的大小已经改变。然而,为了避免被锁定,尽量避免在这里申请(我现在还不太清楚原因;我相信是为了避免在其它线程中调用SDL覆盖函数的原因)。

\begin{lstlisting}
int queue_picture(VideoState *is, AVFrame *pFrame) {

  VideoPicture *vp;
  int dst_pix_fmt;
  AVPicture pict;

  /* wait until we have space for a new pic */
  SDL_LockMutex(is->pictq_mutex);
  while(is->pictq_size >= VIDEO_PICTURE_QUEUE_SIZE &&
    !is->quit) {
    SDL_CondWait(is->pictq_cond, is->pictq_mutex);
  }
  SDL_UnlockMutex(is->pictq_mutex);

  if(is->quit)
    return -1;

  // windex is set to 0 initially
  vp = &is->pictq[is->pictq_windex];

  /* allocate or resize the buffer! */
  if(!vp->bmp ||
     vp->width != is->video_st->codec->width ||
     vp->height != is->video_st->codec->height) {
    SDL_Event event;

    vp->allocated = 0;
    /* we have to do it in the main thread */
    event.type = FF_ALLOC_EVENT;
    event.user.data1 = is;
    SDL_PushEvent(&event);

    /* wait until we have a picture allocated */
    SDL_LockMutex(is->pictq_mutex);
    while(!vp->allocated && !is->quit) {
      SDL_CondWait(is->pictq_cond, is->pictq_mutex);
    }
    SDL_UnlockMutex(is->pictq_mutex);
    if(is->quit) {
      return -1;
    }
  }
\end{lstlisting}

这里的事件机制与前面我们想要退出的时候看到的一样。我们已经定义了事件FF_ALLOC_EVENT 作为SDL_USEREVENT。我们把事件发到事 件队列中然后等待申请内存的函数设置好条件变量。

让我们来看一看如何来修改事件循环:
\begin{lstlisting}
for(;;) {
  SDL_WaitEvent(&event);
  switch(event.type) {
/* ... */
  case FF_ALLOC_EVENT:
    alloc_picture(event.user.data1);
    break;
\end{lstlisting}

记住event.user.data1 是我们的大结构体。就这么简单。让我们看一下alloc_picture()函数:

\begin{lstlisting}
void alloc_picture(void *userdata) {

  VideoState *is = (VideoState *)userdata;
  VideoPicture *vp;

  vp = &is->pictq[is->pictq_windex];
  if(vp->bmp) {
    // we already have one make another, bigger/smaller
    SDL_FreeYUVOverlay(vp->bmp);
  }
  // Allocate a place to put our YUV image on that screen
  vp->bmp = SDL_CreateYUVOverlay(is->video_st->codec->width,
                 is->video_st->codec->height,
                 SDL_YV12_OVERLAY,
                 screen);
  vp->width = is->video_st->codec->width;
  vp->height = is->video_st->codec->height;

  SDL_LockMutex(is->pictq_mutex);
  vp->allocated = 1;
  SDL_CondSignal(is->pictq_cond);
  SDL_UnlockMutex(is->pictq_mutex);
}
\end{lstlisting}

你可以看到我们把SDL_CreateYUVOverlay函数从主循环中移到了这里。这段代码应该完全可以自我注释。记住我们把高度和宽度保存到VideoPicture结构体中因为我们需要保存我们的视频的大小没有因为某些原因而改变。好,我们几乎已经全部解决并且可以申请到YUV覆盖和准备好接收图像。让我们回顾一下queue_picture并看一个拷贝帧到覆盖的代码。你应该能认出其中的一部分:
\begin{lstlisting}
int queue_picture(VideoState *is, AVFrame *pFrame) {

  /* Allocate a frame if we need it... */
  /* ... */
  /* We have a place to put our picture on the queue */

  if(vp->bmp) {

    SDL_LockYUVOverlay(vp->bmp);

    dst_pix_fmt = PIX_FMT_YUV420P;
    /* point pict at the queue */

    pict.data[0] = vp->bmp->pixels[0];
    pict.data[1] = vp->bmp->pixels[2];
    pict.data[2] = vp->bmp->pixels[1];

    pict.linesize[0] = vp->bmp->pitches[0];
    pict.linesize[1] = vp->bmp->pitches[2];
    pict.linesize[2] = vp->bmp->pitches[1];

    // Convert the image into YUV format that SDL uses
    img_convert(&pict, dst_pix_fmt,
        (AVPicture *)pFrame, is->video_st->codec->pix_fmt,
        is->video_st->codec->width, is->video_st->codec->height);

    SDL_UnlockYUVOverlay(vp->bmp);
    /* now we inform our display thread that we have a pic ready */
    if(++is->pictq_windex == VIDEO_PICTURE_QUEUE_SIZE) {
      is->pictq_windex = 0;
    }
    SDL_LockMutex(is->pictq_mutex);
    is->pictq_size++;
    SDL_UnlockMutex(is->pictq_mutex);
  }
  return 0;
}
\end{lstlisting}


这部分代码和前面用到的一样,主要是简单的用我们的帧来填充YUV 覆盖。最后一点只是简单的给队列加1。这个队列在写的时候会一直写入到满为止,在读的时候会一直读空为止。因此所有的都依赖于is->pictq_size值,这要求我们必需要锁定它。这里我们做的是增加写指针(在必要的时候采用轮转的方式),然后锁定队列并且增加尺寸。现在我们的读函数将会知道队列中有了更多的信息,当队列满的时候,我们的写函数也会知道。

\section{显示视频}
上面就是我们的视频线程要做的。现在我们封装了几乎所有的散乱的线程,还剩下一个——记得我们前面调用schedule_refresh()函数吗?让我们看一下实际中是如何做的:

\begin{lstlisting}
/* schedule a video refresh in 'delay' ms */
static void schedule_refresh(VideoState *is, int delay) {
  SDL_AddTimer(delay, sdl_refresh_timer_cb, is);
}
\end{lstlisting}

SDL_AddTimer()是一个SDL函数,按照给定的毫秒来调用(也可以带一些用户数据参数)用户特定的函数。我们将用这个函数来定时刷新视频——每次我们调用这个函数的时候,它将设置一个定时器来触发定时事件来把一帧从图像队列中显示到屏幕上。哈!

但是,让我们先触发那个事件:
\begin{lstlisting}
static Uint32 sdl_refresh_timer_cb(Uint32 interval, void *opaque) {
  SDL_Event event;
  event.type = FF_REFRESH_EVENT;
  event.user.data1 = opaque;
  SDL_PushEvent(&event);
  return 0; /* 0 means stop timer */
}
\end{lstlisting}

这里向队列中写入了一个现在很熟悉的事件。FF_REFRESH_EVENT 被定义成SDL_USEREVENT+1。要注意的一件事是当返回0 的时候,SDL停止定时器,于是回调就不会再发生。

现在我们产生了一个FF_REFRESH_EVENT事件,我们需要在事件循环中处理它:
\begin{lstlisting}
for(;;) {

  SDL_WaitEvent(&event);
  switch(event.type) {
  /* ... */
  case FF_REFRESH_EVENT:
    video_refresh_timer(event.user.data1);
    break;
\end{lstlisting}

于是我们就运行到了这个函数,在这个函数中会把数据从图像队列中取出:
\begin{lstlisting}
void video_refresh_timer(void *userdata) {

  VideoState *is = (VideoState *)userdata;
  VideoPicture *vp;

  if(is->video_st) {
    if(is->pictq_size == 0) {
      schedule_refresh(is, 1);
    } else {
      vp = &is->pictq[is->pictq_rindex];
      /* Timing code goes here */

      schedule_refresh(is, 80);

      /* show the picture! */
      video_display(is);

      /* update queue for next picture! */
      if(++is->pictq_rindex == VIDEO_PICTURE_QUEUE_SIZE) {
        is->pictq_rindex = 0;
      }
      SDL_LockMutex(is->pictq_mutex);
      is->pictq_size--;
      SDL_CondSignal(is->pictq_cond);
      SDL_UnlockMutex(is->pictq_mutex);
    }
  } else {
    schedule_refresh(is, 100);
  }
}
\end{lstlisting}

现在,这只是一个极其简单的函数:当队列中有数据的时候,它从其中获得数据,为下一帧设置定时器,调用video_display 函数来真正显示图像到屏幕上,然后把队列读索引值加1,并且把队列的大小减1。你可能会注意到在这个函数中我们并没有真正对vp 做一些实际的动作,原因是这样的:我们将在后面处理。我们将在后面同步音频和视频的时候用它来访问时间信息。你会在这里看到这个注释信息“timing code here”。那里我们将讨论什么时候显示下一帧视频,然后把相应的值写入到schedule_refresh()函数中。现在我们只是随便写入一个值 80。从技术上来讲,你可以猜测并验证这个值,并且为每个电影重新编译程序,但是:1)过一段时间它会漂移;2)这种方式是很笨的。我们将在后面来讨论它。

我们几乎做完了;我们仅仅剩了最后一件事:显示视频!下面就是video_display函数:
\begin{lstlisting}
void video_display(VideoState *is) {

  SDL_Rect rect;
  VideoPicture *vp;
  AVPicture pict;
  float aspect_ratio;
  int w, h, x, y;
  int i;

  vp = &is->pictq[is->pictq_rindex];
  if(vp->bmp) {
    if(is->video_st->codec->sample_aspect_ratio.num == 0) {
      aspect_ratio = 0;
    } else {
      aspect_ratio = av_q2d(is->video_st->codec->sample_aspect_ratio) *
    is->video_st->codec->width / is->video_st->codec->height;
    }
    if(aspect_ratio <= 0.0) {
      aspect_ratio = (float)is->video_st->codec->width /
    (float)is->video_st->codec->height;
    }
    h = screen->h;
    w = ((int)rint(h * aspect_ratio)) & -3;
    if(w > screen->w) {
      w = screen->w;
      h = ((int)rint(w / aspect_ratio)) & -3;
    }
    x = (screen->w - w) / 2;
    y = (screen->h - h) / 2;

    rect.x = x;
    rect.y = y;
    rect.w = w;
    rect.h = h;
    SDL_DisplayYUVOverlay(vp->bmp, &rect);
  }
}
\end{lstlisting}

因为我们的屏幕可以是任意尺寸(我们设置为640x480并且用户可以自己来改变尺寸),我们需要动态计算出我们显示的图像的矩形大小。所以一开始我们需要计算出电影的\textbf{纵横比}(aspect ratio),表示方式为宽度除以高度。某些编解码器会有奇数\textbf{采样纵横比}(sample aspect ration),只是简单表示了一个像素或者一个采样的宽度除以高度的比例。因为宽度和高度在我们的编解码器中是用像素为单位的,所以实际的纵横比与纵横比乘以采样纵横比相同。某些编解码器会显示纵横比为0,这表示每个像素的纵横比为1x1。然后我们把电影缩放到适合屏幕的尽可能大的尺寸。这里的\& -3表示与-3做与运算,把值舍入到最为接近的4的倍数。\footnote{ 实际上是让它们4 字节对齐。——译者注}然后我们把电影移到中心位置,接着调用SDL_DisplayYUVOverlay() 函数。

结果是什么?我们做完了吗?嗯,我们仍然要重新改写音频部分的代码来使用新的VideoStruct结构体,但是那些改变不大,你可以参考一下示例代码。最后我们要做的是改变ffmpeg提供的默认退出回调函数为我们的退出回调函数。

\begin{lstlisting}
VideoState *global_video_state;

int decode_interrupt_cb(void) {
  return (global_video_state && global_video_state->quit);
}
\end{lstlisting}

我们在主函数中为大结构体设置了global_video_state。

这就好了!让我们编译它:
\begin{lstlisting}
gcc -o tutorial04 tutorial04.c -lavutil -lavformat -lavcodec -lz -lm \
`sdl-config --cflags --libs`
\end{lstlisting}

请享受一下没有经过同步的电影!下次我们将编译一个可以最终工作的电影播放器。

\chapter{同步视频}
\label{ch5}
\section{如何同步视频}
前面整个的一段时间,我们有了一个几乎无用的电影播放器。当然,它能播放视频,也能播放音频,但是它还不能被确切地称之为一部\emph{电影}。那么我们还要做什么呢?
\section{PTS和DTS}
幸运的是,音频和视频流都有一些关于以多快速度和什么时间来播放它们的信息在里面。音频流有采样,视频流有每秒的帧率。然而,如果我们只是简单的通过数帧和乘以帧率的方式来同步视频,那么就很有可能会失去同步。于是作为一种补充,在流中的包有种叫做\textbf{解码时间戳}(DTS)和\textbf{显示时间戳}(PTS)的机制。为了理解这两个参数,你需要了解电影的存储方式。像MPEG等格式,使用被叫做B 帧(B代表“bidrectional”)的方式。另外两种帧被叫做I帧和P帧(I代表"intra",P代表“predicted”)。I帧包含了某个特定的完整图像。P 帧依赖于前面的I帧和P帧并且使用比较或者差分的方式来编码。B帧与P帧有点类似,但是它是依赖于\emph{前面}和\emph{后面}的帧的信息的。这也就解释了为什么我们可能在调用avcodec_decode_video 以后会得不到一帧图像。

所以对于一个电影,帧是这样来显示的:I B B P。现在我们需要在显示B帧之前知道P帧中的信息。因此,帧可能会按照这样的方式来存储:IPBB。这就是为什么我们会有一个解码时间戳和一个显示时间戳的原因。解码时间戳告诉我们什么时候需要解码,显示时间戳告诉我们什么时候需要显示。所以,在这种情况下,我们的流可以是这样的:

\begin{verbatim}
PTS: 1 4 2 3
DTS: 1 2 3 4
Stream: I P B B
\end{verbatim}

通常PTS和DTS只有在流中有B帧的时候会不同。

当我们调用av_read_frame()得到一个包的时候,PTS 和DTS 的信息也会保存在包中。但是我们真正想要的PTS 是我们刚刚解码出来的原始帧的PTS,这样我们才能知道什么时候来显示它。然而,我们从avcodec_decode_video()函数中得到的帧只是一个AVFrame,其中并 没有包含有用的PTS值(注意:AVFrame确实包含一个pts变量,但并不总是我们得到帧的时候想要的值)。然而,ffmpeg 重新排序包以便于被avcodec_decode_video()函数处理的包的DTS可以\emph{总是}与其返回的PTS相同。但是,另外的一个警告是:我们也并不是总能得到这个信息。

不用担心,因为有另外一种办法可以找到帧的PTS,我们可以让程序自己来重新排序包。我们保存一帧的第一个包的PTS:这将作为整个这一帧的PTS。当流没有提供DTS的时候,我们可以使用这个保存下来的PTS。我们可以通过函数avcodec_decode_video()来计算出哪个包是一帧的第一个包。怎样实现呢?任何时候当一个包开始一帧的时候,avcodec_decode_video()将调用一个函数来为一帧申请一个缓冲。当然,ffmpeg 允许我们重新定义那个分配内存的函数。所以我们写了一个新的函数来保存一个包的显示时间戳。

当然,尽管那样,我们可能还是得不到一个正确的时间戳。我们将在后面处理这个问题。

\section{同步}

真不错,现在知道了什么时候来显示一个视频帧,但是我们怎样来实际操作呢?这里有个主意:当我们显示了一帧以后,我们计算出下一帧显示的时间。然后我们简单的设置一个新的定时器用于在那个时间之后刷新视频。正如你可能猜到的,我们检查下一帧的PTS值而不是系统时钟来计算超时时长。这种方式可以工作,但是有两种情况要处理。

首先,要知道下一个PTS是什么时候。现在你应该能想到可以添加视频速率到我们的PTS中——对,很接近了!然而,有些视频需要帧重复。这意味着我们要重复播放当前的帧。这将导致程序显示下一帧太快了。所以我们需要处理它们。

第二,正如程序现在这样,视频和音频播放很欢快,一点也不受同步的影响。如果一切都工作得很好的话,我们不必担心。但是,你的电脑并不是最好的,很多视频文件也不是完好的。所以,我们有三种选择:同步音频到视频,同步视频到音频,或者都同步到外部时钟(例如你的电脑时钟)。从现在开始,我们将同步视频到音频。
\section{编写代码:获得帧的时间戳}
现在让我们到代码中来做这些事情。我们将需要为我们的大结构体添加一些成员,但是我们会只加必要的。先看一下视频线程。记住,在这里我们得到了解码线程输出到队列中的包。这里我们需要的是通过avcodec_decode_video 函数来得到帧的时间戳。我们说的第一种方式是从上次处理的包中得到DTS,这很容易:

\begin{lstlisting}
double pts;

  for(;;) {
    if(packet_queue_get(&is->videoq, packet, 1) < 0) {
      // means we quit getting packets
      break;
    }
    pts = 0;
    // Decode video frame
    len1 = avcodec_decode_video(is->video_st->codec,
                                pFrame, &frameFinished,
                packet->data, packet->size);
    if(packet->dts != AV_NOPTS_VALUE) {
      pts = packet->dts;
    } else {
      pts = 0;
    }
    pts *= av_q2d(is->video_st->time_base);
\end{lstlisting}

如果我们得不到PTS 就把它设置为0。

好,那是很容易的。但是如我们之前所说的,如果包的DTS不能帮到我们,我们需要使用所解码的那一帧的第一个包的PTS。我们通过让ffmpeg 使用我们自己的申请帧程序来实现。下面的是函数的原型:


\begin{lstlisting}
int get_buffer(struct AVCodecContext *c, AVFrame *pic);
void release_buffer(struct AVCodecContext *c, AVFrame *pic);
\end{lstlisting}

申请函数没有告诉我们关于包的任何事情,所以我们要自己每次在得到一个包的时候把PTS保存到一个全局变量中去。我们自己以读到它。然后,我们把值保存到AVFrame结构体的opaque变量中去,这是一个自定义变量,我们可以拿来想干什么就干什么。先来看一下我们的函数:
\begin{lstlisting}
uint64_t global_video_pkt_pts = AV_NOPTS_VALUE;

/* These are called whenever we allocate a frame
 * buffer. We use this to store the global_pts in
 * a frame at the time it is allocated.
 */
int our_get_buffer(struct AVCodecContext *c, AVFrame *pic) {
  int ret = avcodec_default_get_buffer(c, pic);
  uint64_t *pts = av_malloc(sizeof(uint64_t));
  *pts = global_video_pkt_pts;
  pic->opaque = pts;
  return ret;
}
void our_release_buffer(struct AVCodecContext *c, AVFrame *pic) {
  if(pic) av_freep(&pic->opaque);
  avcodec_default_release_buffer(c, pic);
}
\end{lstlisting}

函数avcodec_default_get_buffer 和avcodec_default_release_buffer 是ffmpeg中默认的申请缓冲的函数。函数av_freep是一个内存管理函数,它不但把内存释放而且把指针设置为NULL。

现在到了我们打开流的函数(stream_component_open),我们添加这几行来告诉ffmpeg 如何去做:

\begin{lstlisting}
    codecCtx->get_buffer = our_get_buffer;
    codecCtx->release_buffer = our_release_buffer;
\end{lstlisting}

现在我们必需添加代码来保存PTS到全局变量中,然后在需要的时候来使用它。我们的代码现在看起来应该是这样子:

\begin{lstlisting}
for(;;) {
    if(packet_queue_get(&is->videoq, packet, 1) < 0) {
      // means we quit getting packets
      break;
    }
    pts = 0;

    // Save global pts to be stored in pFrame in first call
    global_video_pkt_pts = packet->pts;
    // Decode video frame
    len1 = avcodec_decode_video(is->video_st->codec, pFrame, &frameFinished,
                packet->data, packet->size);
    if(packet->dts == AV_NOPTS_VALUE
       && pFrame->opaque && *(uint64_t*)pFrame->opaque != AV_NOPTS_VALUE) {
      pts = *(uint64_t *)pFrame->opaque;
    } else if(packet->dts != AV_NOPTS_VALUE) {
      pts = packet->dts;
    } else {
      pts = 0;
    }
    pts *= av_q2d(is->video_st->time_base);
\end{lstlisting}

技术提示:你可能已经注意到我们使用int64 来表示PTS。这是因为PTS是以整型来保存的。这个值是一个时间戳相当于时间的度量,用来以流的time_base为单位进行时间度量。例如,如果一个流是24帧每秒,值为42的PTS 表示这一帧应该排在第42 个帧的位置如果我们每秒有24 帧(这里 并不完全正确)。

我们可以通过除以帧率来把这个值转化为秒。流中的time_base 值表示1/framerate(对于固定帧率来说),所以得到了以秒为单位的PTS,我们需要乘以time_base。

\section{编写代码:使用PTS来同步}

现在我们得到了PTS。我们要注意前面讨论到的两个同步问题。我们将定义一个函数叫做synchronize_video,它可以更新同步的PTS。这个函数也能最终处理我们得不到PTS的情况。同时我们要知道下一帧的时间以便于正确设置刷新速率。我们可以使用内部的反映当前视频已经播放时间的时钟video_clock来完成这个功能。我们把这些值添加到大结构体中。

 \begin{lstlisting}
typedef struct VideoState {
  double          video_clock; //<pts of last decoded frame / predicted pts of next decoded frame
\end{lstlisting}

下面的是函数synchronize_video,它可以很好的自我注释:

\begin{lstlisting}
double synchronize_video(VideoState *is, AVFrame *src_frame, double pts) {

  double frame_delay;

  if(pts != 0) {
    /* if we have pts, set video clock to it */
    is->video_clock = pts;
  } else {
    /* if we aren't given a pts, set it to the clock */
    pts = is->video_clock;
  }
  /* update the video clock */
  frame_delay = av_q2d(is->video_st->codec->time_base);
  /* if we are repeating a frame, adjust clock accordingly */
  frame_delay += src_frame->repeat_pict * (frame_delay * 0.5);
  is->video_clock += frame_delay;
  return pts;
}
\end{lstlisting}

你也许注意到了,我们也计算了重复的帧。

现在让我们得到正确的PTS 并且使用queue_picture来队列化帧,添加一个新的时间戳参数pts:

\begin{lstlisting}
   // Did we get a video frame?
    if(frameFinished) {
      pts = synchronize_video(is, pFrame, pts);
      if(queue_picture(is, pFrame, pts) < 0) {
    break;
      }
    }
\end{lstlisting}

对于queue_picture来说唯一改变的事情就是我们把时间戳值pts保存到VideoPicture结构体中,我们我们必需添加一个时间戳变量到结构体中并且添加一行代码:

\begin{lstlisting}
typedef struct VideoPicture {
  ...
  double pts;
}
int queue_picture(VideoState *is, AVFrame *pFrame, double pts) {
  ... stuff ...
  if(vp->bmp) {
    ... convert picture ...
    vp->pts = pts;
    ... alert queue ...
  }
\end{lstlisting}

现在我们的图像队列中的所有图像都有了正确的时间戳值,所以让我们看一下视频刷新函数。你会记得上次我们用80ms的刷新时间来应付它。那么,现在我们将会算出实际的值。

我们的策略是通过简单计算前一帧和现在这一帧的时间戳来预测出下一个时间戳的时间。同时,我们需要同步视频到音频。我们将设置一个\textbf{音频时钟}(audio clock);一个内部值记录了我们正在播放的音频的位置。就像从任意的mp3播放器中读出来的数字一样。既然我们把视频同步到音频,视频线程使用这个值来算出是否太快还是太慢。

我们将在后面来实现这些代码;现在我们假设我们已经有一个可以给我们音频时钟的函数get_audio_clock。一旦我们有了这个值,我们在音频和视频失去同步的时候应该做些什么呢?简单而有点笨的办法是试着用跳过正确帧或者其它的方式来解决。作为一种替代的手段,我们会调整下次刷新的值;如果时间戳太落后于音频时间,我们加倍计算延迟。如果时间戳太领先于音频时间,我们将尽可能快的刷新。既然我们有了调整过的时间和\textbf{延迟},我们将把它和我们通过frame_timer 计算出来的系统时钟进行比较。这个帧计时器将会统计出电影播放中所有的延时。换句话说,这个 frame_timer 就是指我们什么时候来显示下一帧。我们简单的添加新的帧定时器延时,把它和电脑的系统时间进行比较,然后使用那个值来调度下一次刷新。这可能有点难以理解,所以请认真研究代码:

\begin{lstlisting}
void video_refresh_timer(void *userdata) {

  VideoState *is = (VideoState *)userdata;
  VideoPicture *vp;
  double actual_delay, delay, sync_threshold, ref_clock, diff;

  if(is->video_st) {
    if(is->pictq_size == 0) {
      schedule_refresh(is, 1);
    } else {
      vp = &is->pictq[is->pictq_rindex];

      delay = vp->pts - is->frame_last_pts; /* the pts from last time */
      if(delay <= 0 || delay >= 1.0) {
    /* if incorrect delay, use previous one */
    delay = is->frame_last_delay;
      }
      /* save for next time */
      is->frame_last_delay = delay;
      is->frame_last_pts = vp->pts;

      /* update delay to sync to audio */
      ref_clock = get_audio_clock(is);
      diff = vp->pts - ref_clock;

      /* Skip or repeat the frame. Take delay into account
     FFPlay still doesn't "know if this is the best guess." */
      sync_threshold = (delay > AV_SYNC_THRESHOLD) ? delay : AV_SYNC_THRESHOLD;
      if(fabs(diff) < AV_NOSYNC_THRESHOLD) {
    if(diff <= -sync_threshold) {
      delay = 0;
    } else if(diff >= sync_threshold) {
      delay = 2 * delay;
    }
      }
      is->frame_timer += delay;
      /* computer the REAL delay */
      actual_delay = is->frame_timer - (av_gettime() / 1000000.0);
      if(actual_delay < 0.010) {
    /* Really it should skip the picture instead */
    actual_delay = 0.010;
      }
      schedule_refresh(is, (int)(actual_delay * 1000 + 0.5));
      /* show the picture! */
      video_display(is);

      /* update queue for next picture! */
      if(++is->pictq_rindex == VIDEO_PICTURE_QUEUE_SIZE) {
        is->pictq_rindex = 0;
      }
      SDL_LockMutex(is->pictq_mutex);
      is->pictq_size--;
      SDL_CondSignal(is->pictq_cond);
      SDL_UnlockMutex(is->pictq_mutex);
    }
  } else {
    schedule_refresh(is, 100);
  }
}
\end{lstlisting}

我们在这里做了很多检查:首先,我们保证现在的时间戳和上一个时间戳之间的处以delay是有意义的。如果不是的话,我们就猜测着用上次的延迟。接着,我们有一个同步阈值,因为在同步的时候事情并不总是那么完美的。在ffplay中使用0.01 作为它的值。我们也保证阈值不会比时间戳之间的间隔短。最后,我们把最小的刷新值设置为10毫秒。

\marginpar{\rule[-8mm]{0.4mm}{5mm}}{\textbf{注:}事实上这里我们应该跳过这一帧,但是我们不想为此而烦恼。}

我们给大结构体添加了很多的变量,所以不要忘记检查一下代码。同时也不要忘记在函数streame_component_open中初始化帧时间frame_timer和前面的帧延迟frame delay:

\begin{lstlisting}
  is->frame_timer = (double)av_gettime() / 1000000.0;
  is->frame_last_delay = 40e-3;
\end{lstlisting}

\section{同步:音频时钟}

现在让我们看一下怎样得到音频时钟。我们可以在音频解码函数audio_decode_frame中更新时钟时间。现在,请记住我们并不是每次调用这个函数的时候都在处理新的包,所以有我们要在两个地方更新时钟。第一个地方是我们得到新的包的时候:我们简单的设置音频时钟为这个包的时间戳。然后,如果一个包里有许多帧,我们通过样本数和采样率来计算,所以当我们得到包的时候:

\begin{lstlisting}
   /* if update, update the audio clock w/pts */
    if(pkt->pts != AV_NOPTS_VALUE) {
      is->audio_clock = av_q2d(is->audio_st->time_base)*pkt->pts;
    }
\end{lstlisting}
然后当我们处理这个包的时候:
\begin{lstlisting}
  /* Keep audio_clock up-to-date */
      pts = is->audio_clock;
      *pts_ptr = pts;
      n = 2 * is->audio_st->codec->channels;
      is->audio_clock += (double)data_size /
    (double)(n * is->audio_st->codec->sample_rate);
\end{lstlisting}

一点细节:临时函数被改成包含pts_ptr,所以要保证你已经改了那些。这时的pts_ptr是一个用来通知audio_callback函数当前音频包的时间戳的指针。这将在下次用来同步音频和视频。

现在我们可以最后来实现我们的get_audio_clock 函数。它并不像得到is->audio_clock 值那样简单。注意我们会在每次处理 它的时候设置音频时间戳,但是如果你看了audio_callback 函数,它花费了时间来把数据从音频包中移到我们的输出缓冲区中,这意味着我们音频时钟中记录的时间比实际的要早太多。所以我们必须要检查一下我们还有多少没有写入。下面是完整的代码:

\begin{lstlisting}
double get_audio_clock(VideoState *is) {
  double pts;
  int hw_buf_size, bytes_per_sec, n;

  pts = is->audio_clock; /* maintained in the audio thread */
  hw_buf_size = is->audio_buf_size - is->audio_buf_index;
  bytes_per_sec = 0;
  n = is->audio_st->codec->channels * 2;
  if(is->audio_st) {
    bytes_per_sec = is->audio_st->codec->sample_rate * n;
  }
  if(bytes_per_sec) {
    pts -= (double)hw_buf_size / bytes_per_sec;
  }
  return pts;
}
\end{lstlisting}

你应该知道为什么这个函数可以正常工作了;)

这就是了!让我们编译它:

\begin{lstlisting}
gcc -o tutorial05 tutorial05.c -lavutil -lavformat -lavcodec -lz
-lm`sdl-config --cflags --libs`
\end{lstlisting}

最后,你可以使用我们自己的电影播放器来看电影了。下次我们将看一下音频同步,然后接下来的教程我们会讨论定位(seeking)。

\chapter{同步音频}
\label{ch6}
\section{同步音频}
现在我们已经有了一个比较像样的播放器。所以让我们看一下还有哪些零碎的东西没处理。上次,我们掩饰了一点同步问题,也就是同步音频到视频时钟而不是相反。我们将采用和视频一样的方式:做一个内部视频时钟来记录视频线程播放了多久,然后同步音频到上面去。后面我们也来看一下如何推而广之把音频和视频都同步到外部时钟。
\section{实现视频时钟}
现在我们要生成一个类似于上次我们的音频时钟的视频时钟:一个给出当前视频播放时间的内部值。开始,你可能会想这和使用上一帧的时间戳来更新定时器一样简单。但是,不要忘了视频帧之间的时间间隔是很长的,以毫秒为计量的。解决办法是跟踪另外一个值:我们将视频时钟设置为上一帧时间戳的时候的时间值。于是当前视频时间值就是PTS_of_last_frame + (current_time -time_elapsed_since_PTS_value_was_set)。这种解决方式与我们在函数get_audio_clock 中的方式很类似。

所在在我们的大结构体中,我们将放上一个双精度浮点变量video_current_pts和一个64位宽整型变量video_current_pts_time。时钟更新将被放在video_refresh_timer函数中。

\begin{lstlisting}
void video_refresh_timer(void *userdata) {

  /* ... */

  if(is->video_st) {
    if(is->pictq_size == 0) {
      schedule_refresh(is, 1);
    } else {
      vp = &is->pictq[is->pictq_rindex];

      is->video_current_pts = vp->pts;
      is->video_current_pts_time = av_gettime();
\end{lstlisting}
不要忘记在stream_component_open 函数中初始化它:
\begin{lstlisting}
  is->video_current_pts_time = av_gettime();
\end{lstlisting}
现在我们需要一种得到信息的方式:
\begin{lstlisting}
double get_video_clock(VideoState *is) {
  double delta;

  delta = (av_gettime() - is->video_current_pts_time) / 1000000.0;
  return is->video_current_pts + delta;
}
\end{lstlisting}
\section{提取时钟}
但是为什么要强制使用视频时钟呢?以致于我们不得不更改视频同步代码来让音频和视频不会去相互同步。想像一下如果我们让它像ffplay 一样有一个命令行参数得有多混乱。所以让我们概括一下要做的事情:我们将做一个新的封装函数get_master_clock,用来检测av_sync_type 变量然后决定调用get_audio_clock还是get_video_clock 或者其它的想使用的获得时钟的函数。我们甚至可以使用电脑时钟,通过调用get_external_clock函数:

\begin{lstlisting}
enum {
  AV_SYNC_AUDIO_MASTER,
  AV_SYNC_VIDEO_MASTER,
  AV_SYNC_EXTERNAL_MASTER,
};

#define DEFAULT_AV_SYNC_TYPE AV_SYNC_VIDEO_MASTER

double get_master_clock(VideoState *is) {
  if(is->av_sync_type == AV_SYNC_VIDEO_MASTER) {
    return get_video_clock(is);
  } else if(is->av_sync_type == AV_SYNC_AUDIO_MASTER) {
    return get_audio_clock(is);
  } else {
    return get_external_clock(is);
  }
}
main() {
...
  is->av_sync_type = DEFAULT_AV_SYNC_TYPE;
...
}
\end{lstlisting}

\section{同步音频}

现在是最难的部分:同步音频到视频时钟。我们的策略是测量音频的位置,把它与视频时间比较然后算出我们需要修正多少的样本数,也就是说:我们是否需要通过丢弃样本的方式来加速播放还是需要通过插值样本的方式来放慢播放?

我们将在每次处理音频样本的时候运行一个synchronize_audio的函数来正确的收缩或者扩展音频样本。然而,我们不想在每次发现有偏差的时候都进行同步,因为这样会使同步音频多于视频包。所以我们为函数synchronize_audio设置一个最小连续值来限定需要同步的时刻,这样我们就不会总是在调整了。当然,就像上次那样,“失去同步”意味着音频时钟和视频时钟的差异大于我们的阈值。

所以我们将使用一个分数系数,叫c,所以现在可以说我们得到了N个失去同步的音频样本。失去同步的数量可能会有很多变化,所以我们要计算一下失去同步的长度的均值。例如,第一次调用的时候,显示出来我们失去同步的长度为40ms,下次变为50ms等等。但是我们不会使用一个简单的均值,因为距离现在最近的值比靠前的值要重要的多。所以我们将使用一个分数系数,叫c,然后用这样的公式来计算差异:diff_sum = new_diff + diff_sum*c。当我们准备好去找平均差异的时候,我们用简单的计算方式:avg_diff = diff_sum * (1-c)。

\marginpar{\rule[-15mm]{0.4mm}{15mm}}{\small
\textbf{注意:}到底是怎么回事?这个公式看来很神奇!嗯,它基本上是一个使用等比级数的加权平均值。我不知道这是否有名字(我甚至查过维基百科!),但是如果想要更多的信息,\href{http://www.dranger.com/ffmpeg/weightedmean.html}{这里是一个解释}(或者\href{http://www.dranger.com/ffmpeg/weightedmean.txt}{weightedmean.txt} )。}

下面是到目前为止我们的函数:
\begin{lstlisting}
/* Add or subtract samples to get a better sync, return new
   audio buffer size */
int synchronize_audio(VideoState *is, short *samples,
              int samples_size, double pts) {
  int n;
  double ref_clock;

  n = 2 * is->audio_st->codec->channels;

  if(is->av_sync_type != AV_SYNC_AUDIO_MASTER) {
    double diff, avg_diff;
    int wanted_size, min_size, max_size, nb_samples;

    ref_clock = get_master_clock(is);
    diff = get_audio_clock(is) - ref_clock;

    if(diff < AV_NOSYNC_THRESHOLD) {
      // accumulate the diffs
      is->audio_diff_cum = diff + is->audio_diff_avg_coef
    * is->audio_diff_cum;
      if(is->audio_diff_avg_count < AUDIO_DIFF_AVG_NB) {
    is->audio_diff_avg_count++;
      } else {
    avg_diff = is->audio_diff_cum * (1.0 - is->audio_diff_avg_coef);

       /* Shrinking/expanding buffer code.... */

      }
    } else {
      /* difference is TOO big; reset diff stuff */
      is->audio_diff_avg_count = 0;
      is->audio_diff_cum = 0;
    }
  }
  return samples_size;
}
\end{lstlisting}

现在我们已经做得很好了;我们已经近似地知道如何用视频或者其它的时钟来调整音频了。所以让我们来计算一下要再添加和丢掉多少样本,把下面代码放到注释“Shrinking/expanding buffer code”的地方:

\begin{lstlisting}
if(fabs(avg_diff) >= is->audio_diff_threshold) {
  wanted_size = samples_size +
  ((int)(diff * is->audio_st->codec->sample_rate) * n);
  min_size = samples_size * ((100 - SAMPLE_CORRECTION_PERCENT_MAX)
                             / 100);
  max_size = samples_size * ((100 + SAMPLE_CORRECTION_PERCENT_MAX)
                             / 100);
  if(wanted_size < min_size) {
    wanted_size = min_size;
  } else if (wanted_size > max_size) {
    wanted_size = max_size;
  }
\end{lstlisting}

记住audio_length * (sample_rate * \# of channels * 2)就是audio_length秒长的音频的样本数。所以,我们想要的样本数就是我们根据音频偏移添加或者减少后的音频样本数。我们也可以设置一个范围来限定我们一次进行修正的长度,因为如果我们改变的太多,用户会听到刺耳的音频。

\section{修正样本数}

现在我们要真正的修正一下音频。你可能会注意到我们的同步函数synchronize_audio返回了一个样本数,这可以告诉我们有多少个字节被送到流中。所以我们只要调整样本数为wanted_size就可以了。这会让样本更小一些。但是如果我们想让它变大,我们不能只是让样本大小变大,因为在缓冲区中没有多余的数据!所以我们必需添加上去。但是我们怎样来添加呢?最笨的办法就是试着来外插音频,我们使用已经在缓冲区的数据,并把最后几个样本再填充到缓冲区。

\begin{lstlisting}
if(wanted_size < samples_size) {
  /* remove samples */
  samples_size = wanted_size;
} else if(wanted_size > samples_size) {
  uint8_t *samples_end, *q;
  int nb;

  /* add samples by copying final samples */
  nb = (samples_size - wanted_size);
  samples_end = (uint8_t *)samples + samples_size - n;
  q = samples_end + n;
  while(nb > 0) {
    memcpy(q, samples_end, n);
    q += n;
    nb -= n;
  }
  samples_size = wanted_size;
}
\end{lstlisting}

函数返回的是样本数,这个函数就完成了。现在要做的是使用它:

\begin{lstlisting}
void audio_callback(void *userdata, Uint8 *stream, int len) {

  VideoState *is = (VideoState *)userdata;
  int len1, audio_size;
  double pts;

  while(len > 0) {
    if(is->audio_buf_index >= is->audio_buf_size) {
      /* We have already sent all our data; get more */
      audio_size = audio_decode_frame(is, is->audio_buf, sizeof(is->audio_buf), &pts);
      if(audio_size < 0) {
    /* If error, output silence */
    is->audio_buf_size = 1024;
    memset(is->audio_buf, 0, is->audio_buf_size);
      } else {
    audio_size = synchronize_audio(is, (int16_t *)is->audio_buf,
                       audio_size, pts);
    is->audio_buf_size = audio_size;
\end{lstlisting}

我们要做的是把函数synchronize_audio 插入进去。(同时,保证在初始化上面
变量的时候检查一下代码,这些我没有赘述)。

结束之前的最后一件事情:我们需要添加一个if 语句来保证我们不会在视频为
主时钟的时候也来同步视频。
\begin{lstlisting}
if(is->av_sync_type != AV_SYNC_VIDEO_MASTER) {
  ref_clock = get_master_clock(is);
  diff = vp->pts - ref_clock;

  /* Skip or repeat the frame. Take delay into account
     FFPlay still doesn't "know if this is the best guess." */
  sync_threshold = (delay > AV_SYNC_THRESHOLD) ? delay :
                    AV_SYNC_THRESHOLD;
  if(fabs(diff) < AV_NOSYNC_THRESHOLD) {
    if(diff <= -sync_threshold) {
      delay = 0;
    } else if(diff >= sync_threshold) {
      delay = 2 * delay;
    }
  }
}
\end{lstlisting}

添加后就可以了。要保证整个程序中我没有赘述的变量都被初始化过了。然后编译它:
\begin{lstlisting}
gcc -o tutorial06 tutorial06.c -lavutil -lavformat -lavcodec -lz -lm`sdl-config --cflags --libs`
\end{lstlisting}
然后你就可以运行它了。

下次我们要做的是让你可以让电影快退和快进。

\chapter{快进快退}
\label{ch7}
\section{处理快进快退(seek)命令}
现在我们来为我们的播放器加入一些快进和快退的功能,因为如果你不能在一部电影中来回跳转(rewind)是很让人讨厌的。同时,这将告诉你av_seek_frame 函数是多么容易使用。

我们将在电影播放中使用左方向键和右方向键来表示向后和向前一小段,使用向上和向下键来表示向前和向后一大段。这里一小段是10 秒,一大段是60 秒。所以我们需要设置我们的主循环来捕捉键盘事件。然而当我们捕捉到键盘事件后我们不能直接调用av_seek_frame 函数。我们要在主要的解码循环,decode_thread循环中做这些。所以,我们要添加一些变量到大结构体中,用来包含新的跳转位置和一些跳转标志:
\begin{lstlisting}
  int             seek_req;
  int             seek_flags;
  int64_t         seek_pos;
\end{lstlisting}

现在让我们在主循环中捕捉按键:

\begin{lstlisting}
for(;;) {
    double incr, pos;

    SDL_WaitEvent(&event);
    switch(event.type) {
    case SDL_KEYDOWN:
      switch(event.key.keysym.sym) {
      case SDLK_LEFT:
    incr = -10.0;
    goto do_seek;
      case SDLK_RIGHT:
    incr = 10.0;
    goto do_seek;
      case SDLK_UP:
    incr = 60.0;
    goto do_seek;
      case SDLK_DOWN:
    incr = -60.0;
    goto do_seek;
      do_seek:
    if(global_video_state) {
      pos = get_master_clock(global_video_state);
      pos += incr;
      stream_seek(global_video_state,
                      (int64_t)(pos * AV_TIME_BASE), incr);
    }
    break;
      default:
    break;
      }
      break;
\end{lstlisting}

为了检测按键,我们先查了一下是否有SDL_KEYDOWN事件。然后我们使用event.key.keysym.sym来判断哪个按键被按下。一旦我们知道了如何来跳转,我们就来计算新的时间,方法为把增加的时间值加到从函数get_master_clock中得到的时间值上。然后我们调用stream_seek 函数来设置seek_pos等变量。我们把新的时间转换成为avcodec中的内部时间戳单位。在流中调用那个时间戳将使用帧而不是用秒来计算,公式为seconds = frames * time_base(fps)。默认的avcodec值为1,000,000fps(所以2 秒的内部时间戳为2,000,000)。我们后面再来看一下为什么要把这个值进行一下转换。

这就是我们的stream_seek函数。请注意我们设置了一个标志为后退服务:

\begin{lstlisting}
void stream_seek(VideoState *is, int64_t pos, int rel) {

  if(!is->seek_req) {
    is->seek_pos = pos;
    is->seek_flags = rel < 0 ? AVSEEK_FLAG_BACKWARD : 0;
    is->seek_req = 1;
  }
}
\end{lstlisting}

现在让我们回到实现跳转的decode_thread函数。你会注意到我们已经在源文件中标记了一个叫做“seek stuff goes here”的部分。现在我们将把代码写在这里。

跳转是围绕着av_seek_frame函数实现的。这个函数用到了一个格式上下文,一个流,一个时间戳和一组标记来作为它的参数。这个函数将会跳转到你所给的时间戳的位置。时间戳的单位是你传递给函数的流的time_base。然而,你并不是必需要传给它一个流(流可以用-1 来代替)。如果你这样做了,time_base将会是avcodec中的内部时间戳单位,或者是1000000fps。这就是为什么我们在设置seek_pos 的时候会把位置乘以AV_TIME_BASER的原因。

但是,如果给av_seek_frame 函数的stream 参数传递传-1,你有时会在播放某些文件的时候遇到问题(比较少见),所以我们会取文件中的第一个流并且把它传递到av_seek_frame函数。不要忘记我们也要把时间戳timestamp的单位进行转化。

\begin{lstlisting}
if(is->seek_req) {
  int stream_index= -1;
  int64_t seek_target = is->seek_pos;

  if     (is->videoStream >= 0) stream_index = is->videoStream;
  else if(is->audioStream >= 0) stream_index = is->audioStream;

  if(stream_index>=0){
    seek_target= av_rescale_q(seek_target, AV_TIME_BASE_Q,
                      pFormatCtx->streams[stream_index]->time_base);
  }
  if(av_seek_frame(is->pFormatCtx, stream_index,
                    seek_target, is->seek_flags) < 0) {
    fprintf(stderr, "%s: error while seeking\n",
            is->pFormatCtx->filename);
  } else {
     /* handle packet queues... more later... */
\end{lstlisting}

这里av_rescale_q(a,b,c)是用来把时间戳从一个时基调整到另外一个时基时候用的函数。它基本的动作是计算a*b/c,但是这个函数还是必需的,因为直接计算会有溢出的情况发生。AV_TIME_BASE_Q是AV_TIME_BASE 作为分母后的版本。它们是很不相同的:AV_TIME_BASE * time_in_seconds = avcodec_timestamp而AV_TIME_BASE_Q * avcodec_timestamp = time_in_seconds(注意AV_TIME_BASE_Q实际上是一个AVRational对象,所以你必需使用avcodec中特定的q 函数 来处理它)。

\section{清空我们的缓冲}

我们已经正确设定了跳转位置,但是我们还没有结束。记住我们有一个放了很多包的队列。既然我们跳到了不同的位置,我们必需把队列中的内容清空否则电影是不会跳转的。不仅如此,avcodec也有它自己的内部缓冲,也需要每次被清空。

要实现这个,我们需要首先写一个函数来清空我们的包队列。然后我们需要一种通知音频和视频线程来清空avcodec内部缓冲的办法。我们可以在清空队列后把特定的包放入到队列中,然后当它们检测到特定的包的时候,它们就会把自己的内部缓冲清空。让我们开始写清空函数。其实很简单的,所以我直接把代码写在下面:
\begin{lstlisting}
static void packet_queue_flush(PacketQueue *q) {
  AVPacketList *pkt, *pkt1;

  SDL_LockMutex(q->mutex);
  for(pkt = q->first_pkt; pkt != NULL; pkt = pkt1) {
    pkt1 = pkt->next;
    av_free_packet(&pkt->pkt);
    av_freep(&pkt);
  }
  q->last_pkt = NULL;
  q->first_pkt = NULL;
  q->nb_packets = 0;
  q->size = 0;
  SDL_UnlockMutex(q->mutex);
}
\end{lstlisting}

既然队列已经清空了,我们放入“清空包”。但是开始我们要定义和创建这个包:

\begin{lstlisting}
AVPacket flush_pkt;

main() {
  ...
  av_init_packet(&flush_pkt);
  flush_pkt.data = "FLUSH";
  ...
}
\end{lstlisting}

现在我们把这个包放到队列中:

\begin{lstlisting}
  } else {
    if(is->audioStream >= 0) {
      packet_queue_flush(&is->audioq);
      packet_queue_put(&is->audioq, &flush_pkt);
    }
    if(is->videoStream >= 0) {
      packet_queue_flush(&is->videoq);
      packet_queue_put(&is->videoq, &flush_pkt);
    }
  }
  is->seek_req = 0;
}
\end{lstlisting}

(这些代码片段是接着前面decode_thread 中的代码片段的)我们也需要修改packet_queue_put 函数才不至于直接简单复制了这个包:

\begin{lstlisting}
int packet_queue_put(PacketQueue *q, AVPacket *pkt) {

  AVPacketList *pkt1;
  if(pkt != &flush_pkt && av_dup_packet(pkt) < 0) {
    return -1;
  }
\end{lstlisting}

然后在音频线程和视频线程中,我们在packet_queue_get后立即调用函数:

\begin{lstlisting}
   if(packet_queue_get(&is->audioq, pkt, 1) < 0) {
      return -1;
    }
    if(packet->data == flush_pkt.data) {
      avcodec_flush_buffers(is->audio_st->codec);
      continue;
    }
\end{lstlisting}

上面的代码片段与视频线程中的一样,只要把“audio”换成“video”。

就这样,编译我们的播放器:

\begin{lstlisting}
gcc -o tutorial07 tutorial07.c -lavutil -lavformat -lavcodec -lz -lm`sdl-config --cflags --libs`
\end{lstlisting}

试一下!我们几乎已经都做完了;下次我们只要做一点小的改动就好了,那就是看看ffmpeg 提供的小的软件缩放采样。


\chapter{软件缩放}
\label{ch8}
\section{软件缩放库libswscale}
近来ffmpeg添加了新的接口,使用libswscale来处理图像缩放。既然在前面我们使用img_convert来把RGB转换成YUV12,那么我们现在来使用新的接口。新接口更加标准和快速,而且我相信里面有了MMX优化代码。换句话说,它是做缩放更好的方式。

我们将用来缩放的基本函数是sws_scale。但一开始,我们必需建立一个SwsContext的结构体。这将让我们进行想要的转换,然后把它传递给sws_scale函数。类似于在SQL 中的预备阶段或者是在Python中编译的规则表达式regexp。要准备这个上下文,我们使用sws_getContext 函数,它需要我们源的宽度和高度,我们想要的宽度和高度,源的格式和想要转换成的格式,同时还有一些其它的参数和标志。然后我们像使用img_convert一样来使用sws_scale函数,唯一不同的是我们传递给的是SwsContext:
\begin{lstlisting}
#include <ffmpeg/swscale.h> // include the header!

int queue_picture(VideoState *is, AVFrame *pFrame, double pts) {

  static struct SwsContext *img_convert_ctx;
  ...

  if(vp->bmp) {

    SDL_LockYUVOverlay(vp->bmp);

    dst_pix_fmt = PIX_FMT_YUV420P;
    /* point pict at the queue */

    pict.data[0] = vp->bmp->pixels[0];
    pict.data[1] = vp->bmp->pixels[2];
    pict.data[2] = vp->bmp->pixels[1];

    pict.linesize[0] = vp->bmp->pitches[0];
    pict.linesize[1] = vp->bmp->pitches[2];
    pict.linesize[2] = vp->bmp->pitches[1];

    // Convert the image into YUV format that SDL uses
    if(img_convert_ctx == NULL) {
      int w = is->video_st->codec->width;
      int h = is->video_st->codec->height;
      img_convert_ctx = sws_getContext(w, h,
                        is->video_st->codec->pix_fmt,
                        w, h, dst_pix_fmt, SWS_BICUBIC,
                        NULL, NULL, NULL);
      if(img_convert_ctx == NULL) {
    fprintf(stderr, "Cannot initialize the conversion context!\n");
    exit(1);
      }
    }
    sws_scale(img_convert_ctx, pFrame->data,
              pFrame->linesize, 0,
              is->video_st->codec->height,
              pict.data, pict.linesize);
\end{lstlisting}

我们把新的缩放器放到了合适的位置。希望这会让你知道libswscale 能做什么。

就这样,我们做完了!编译我们的播放器:
\begin{lstlisting}
gcc -o tutorial08 tutorial08.c -lavutil -lavformat -lavcodec -lz -lm `sdl-config --cflags --libs`
\end{lstlisting}
享受我们用C写的少于1000行的电影播放器吧。

当然,也还是有很多我们没太关注的东西可以添加进来的。

\chapter*{现在还要做什么?}
\label{ch9}
我们已经有了一个可以工作的播放器,当然,还还不够好。我们做了很多,但是还有很多可以添加的功能:
\begin{itemize}
\item 错误处理。我们代码中的错误处理是糟透的,多处理一些会更好。
\item 暂停。我们不能暂停电影,这是一个很有用的功能。我们可以在大结构体中使用一个内部暂停变量,当用户暂停的时候就设置它。然后我们的音频,视频和解码线程检测到它后就不再输出任何东西。我们也使用av_read_play 来支持网络。这很容易解释,但是你自己来做的话不那么容易,所以如果你想尝试的话,\textbf{把这个作为一个家庭作业}。提示,可以参考ffplay.c。
\item 支持视频硬件特性。一个参考的例子,请参考\href{http://www.inb.uni-luebeck.de/~boehme/libavcodec_update.html}{Martin的旧教程}中的Frame Grabbing 部分。

\item 按字节跳转。如果你可以按照字节而不是秒的方式来计算出跳转位置,那么对于像VOB 文件一样的有不连续时间戳的视频文件来说,定位会更加精确。
\item 丢弃帧。如果视频落后的太多,我们应当把下一帧丢弃掉而不是设置一个短的刷新时间。
\item 支持网络。现在的电影播放器还不能播放网络流媒体。
\item 支持像YUV 文件一样的原始视频流。如果我们的播放器支持的话,因为我们不能猜测出时基和大小,我们应该加入一些参数来进行相应的设置。
\item 全屏。
\item 各种选项,例如:不同图像格式;参考ffplay.c 中的命令开关。
\item 其它事情,例如:在结构体中的音频缓冲区应该对齐。
\end{itemize}

我们目前仅仅涵盖了ffmpeg中的一小部分,如果你想了解关于ffmpeg 更多的事情,下一步应该学习的就是如何来\emph{编码}多媒体。一个好的入手点是在 ffmpeg 中的output_example.c 文件。我可以为它写另外一个教程,但是我没有足够的时间来做。

好,我希望这个教程是有益和有趣的。如果你有任何建议,问题,抱怨和赞美等,请给发邮件到dranger at gmail dot com。




%\backmatter

\end{document}
